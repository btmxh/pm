\chapter{Differentiation} % (fold)
\label{chap:Differentiation}

\section{Exercises 4A} % (fold)
\label{sec:Exercises 4A}
\begin{enumerate}[label=\textbf{4A.\arabic*}]
  \item This trivially follows from:
    \[
      \int |h|^{p}d\mu \ge \int _{|h(x)| \ge c} |h|^{p}d\mu \ge c^{p}\mu (\{x \in
      X: |h(x)|\ge c\}  )
    .\] 
  \item Let \( u = \int hd\mu \) and \( f = h - u \). Then, by Markov's
    inequality, we have:
    \begin{align*}
      \mu \left( \left\{ x \in X: \left| h(x)-\int hd\mu \right| \ge c  \right\}
      \right) &= \mu (\{x \in X: |f(x)| \ge c \}  ) \\
              &\le \frac{1}{c ^2}\int f^2d\mu\\
              &= \frac{1}{c ^2} \int (h - u)^2d\mu \\
              &= \frac{1}{c ^2} \left( \int h^2d\mu - 2u\int hd\mu + u^2\mu (X) \right)  \\
              &= \frac{1}{c ^2} \left( \int h^2d\mu - \left( \int hd\mu  \right)
              ^2\right)
    .\end{align*}
  \item Equality in Markov's inequality occurs if and only if \( \int
    _{|h(x)|<c} hd\mu = 0 \) and \( h(x) = c, \forall x \in X, h(x) \ge c \).
    The latter implies that if \( c_{1}, c_{2} \) both satisfies the given
    equality, then \( c_{1} = c_{2} \).
  \item Consider the two intervals \( I_{1} = (0, 1 + \varepsilon) \) and \(
    I_{2} = (1 - \varepsilon, 2) \), then since they intersects for all \(
    \varepsilon > 0 \), if the Lemma holds for constant \( c \), then \( c \star
    I_{1}\) must contain \( I_{2} \) or \( c \star I_{2} \) must contain \(
    I_{1} \). Either of those can only happens if \( c \ge
    \frac{3-\varepsilon}{1+\varepsilon} \). Letting \( \varepsilon \to 0 \)
    yields \( c \ge 3 \).
  \item Trivial, so no.
  \item If \( b \in (0, 1) \), then \( \chi_{[0, 1]}^{*}(b) = \sup _{t > 0}
    \frac{1}{2t} \int _{b-t}^{b+t} \chi_{[0, 1]} \). We have \( \int
    _{b-t}^{b+t} \chi_{[0, 1]} \le \int _{b-t}^{b+t} 1 = 2t \), so the supremum
    is at least \( 1 \). Equality does happen when one picks \( t < \min \{b,
    1-b\}   \).

    If \( b \le  0 \), then the \( \int _{b-t}^{b+t}\chi_{[0, 1]} \) is non-zero
    when \( b + t > 0 \), and the integral is equal to \( \int _{0}^{\min \{b+t,
    1\}  } 1 = \min \{b+t, 1\}   \). Hence, \( \chi_{[0, 1]}^{*}(b) = \sup _{t >
    -b} \frac{\min \{b+t, 1\}  }{2t} = \frac{1}{2(1-b)} \).

    If \( b \ge  1 \), one proceed similarly to the \( b \le 0 \) case.
  \item 
    We can directly calculate \( \chi_{[0,1] \cup [2, 3]}^{*}(b) \) as follows:
    \begin{align*}
      \chi_{[0,1]\cup [2,3]}^{*}(b) &= \sup _{t > 0} \frac{1}{2t} \int
      \chi_{[b-t, b+t]}(\chi_{[0,1]} + \chi_{[2,3]}) d\lambda\\
      &= \sup _{t>0} \frac{1}{2t} (\lambda([b-t,b+t]\cap [0,1]) +
      \lambda([b-t,b+t] \cap [2,3])) \\
      &= \sup _{t > 0} \frac{1}{2t} \left( \min \{1, b+t\} + \min \{3, b+t\} -
      \max \{0, b-t\} - \max\{2, b-t\}    \right) 
    .\end{align*}
    Abusing graphing calculator, we have the following table:
    \[
      \begin{array}{c|c|c}
        b & t & \chi^{*}\\
        \hline
        (-\infty, 0] & 3 - b & \frac{1}{3-b}\\
        \hline
        (0, 1) & \varepsilon \to 0 & 1\\
        \hline
        [1, 2] & b & \min \{\frac{1}{b}, 1 - \frac{1}{2b}\}\\
        \hline
        (2, 3) & \varepsilon \to 0 & 1\\
        \hline
        [3, +\infty) & b & \frac{1}{b}
      \end{array}
    .\]
  \item We have:
    \[
      h^{*}(b)= \sup _{t > 0} \frac{1}{2t} \int _{b-t}^{b+t} \chi_{[0,1]}x =
      \sup _{t \in T} \frac{1}{2t} \int _{\max \{0, b-t\}  }^{\min \{1, b+t\}  } xdx =
      \sup _{t \in T} \frac{1}{4t}\left( \min \{1, b + t\} ^2 - \max \{0, b - t\}
      ^2   \right) \le b
    ,\] with \( T = \{t > 0: \min \{1, b + t\} \ge \max \{0, b - t\}   \}   \).

    Equality occurs when \( 0 < b - t < b + t < 1 \), or \( b \in (0, 1) \).

    If \( b \le  0 \), then \( \max \{0, b-t\} = 0, \forall t > 0  \) and \( T =
    \{t > 0: b + t > 0\} = (-b, \infty)     \), so \(
    h^{*}(b) = \sup _{t > -b} \frac{\min \{1, b+t\} ^2 }{4t} = \frac{1}{4(1-b)} \)

    If \( b \ge  1 \), then \( \min \{1, b + t\} = 1, \forall t > 0     \) and \( T
    = \{t > 0: 1 > b - t\} = (b-1, +\infty)    \), so \( h^{*}(b) = \sup _{t >
    b-1} \frac{1 - \max \{0, b-t\}^2  }{4t} = \frac{b - \sqrt{b^2-1} }{2} \).

\end{enumerate}
% section Exercises 4A (end)

% chapter Differentiation (end)
