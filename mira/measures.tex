\chapter{Measures} % (fold)
\label{chap:Measures}

\section{Exercises 2A} % (fold)
\label{sec:Exercises 2A}

\begin{enumerate}[label=\textbf{2A.\arabic*}]
  \item \label{2A1}
    This trivially follows from subadditivity:
    \[
      |A| \le |A \cup B| \le |A| + |B| = |A|
    .\] 
  \item If \( t = 0 \), then we have \( |0A| = 0 = 0 |A| \). Otherwise, we have:
    \begin{align*}
      |tA| &= \inf \{\sum_{k = 1}^{+\infty} \ell(I_{k}), I_{k \in
      1..+\infty} \text{interval covers} tA\}  \\
           &= |t|\inf \{\sum_{k = 1}^{+\infty} \ell\left( \frac{1}{t} I_{k} \right) , \frac{1}{t}I_{k \in 1..+\infty} \text{interval covers} A\}  \\
           &= |t| |A|
    .\end{align*}
  \item This trivially follows from subadditivity: \( |A| + |B \setminus A| \ge
    |B|\).
  \item \textbf{Bounded}: Consider the open cover \( I_{k} = (-k, k) \) of \( \mathbb{R} \) (and
    therefore \( F \)). Since this cover has a finite subcover for \( F \),
    denote that subcover as \( I_{k_{1}}, I_{k_{2}}, \ldots , I_{k_{n}} \), we
    can easily see that \( F \subseteq I_{K}\), with \( K = \max_{i \in 1..n}
    k_{i} \). Hence, \( F \) is bounded.

    \textbf{Closed}: Let \( x_{0} \in F^{c} \), we will prove that there exists
    some \( \varepsilon > 0 \) such that \( B(x_{0}, \varepsilon) = (x_{0} -
    \varepsilon, x_{0} + \varepsilon) \subseteq F^{c} \).

    Consider the open cover \( I(x \in F) = B(x, \varepsilon(x)) \) of \( F \),
    with \( \varepsilon(x) \) being a positive constant depending on \( x \).
    Since \( F \) is compact, there exists finitely many \( x_{1}, x_{2}, \ldots
    , x_{n}\) such that \( F \subseteq B(x_{1}, \varepsilon(x_{1})) \cup B(x_{2},
    \varepsilon(x_{2})) \cup  \ldots  \cup B(x_{n}, \varepsilon(x_{n})) \).

    Now, let \( d \) the minimum distance\footnote{The distance between two
    real numbers \( a \) and \( b \) is defined as \( d(a, b) = |a - b| \)} of
    \( x_{0} \) to the \( x_{i} \)'s.
    Consider a point \( x \in F \), then there exists an \( i \in 1..n \) such
    that: \( x \in B(x_{i}, \varepsilon(x_{i})) \) or \( d(x, x_{i}) <
    \varepsilon(x_{i}) \). Then, by the triangle inequality, we have:
    \[
      d(x_{0}, x) \ge d(x_{0}, x_{i}) - d(x, x_{i}) > d(x_{0}, x_{i}) -
      \varepsilon(x_{i})
    .\] 
    Now, if we pick \( \varepsilon(x_{i}) > 0 \) such that RHS is always greater
    than \( 0 \), for example \( \varepsilon(x_{i}) = \frac{1}{2}d(x_{0}, x_{i})
    \) (note that we obviously have \( x_{0} \neq  x_{i} \) for all \( i \in
    1..n \)), then we have:
    \[
      d(x_{0}, x) > \min_{i \in 1..n} \{d - \varepsilon(x_{i})\} = r > 0, \forall x \in
      F
    .\]
    From this, we see that \( B(x, r) \subseteq F^{c} \), therefore \( F^{c} \)
    is open and \( F \) is closed.
  \item \label{2A5} Let \( B \) be the bounded set in \( \mathcal{A} \), then we have:
    \begin{align*}
      \varnothing =  \bigcap_{F \in \mathcal{A}} F &= B \cap \left(
      \bigcap_{F \in \mathcal{A} \setminus \{B\}  } F \right)  \\
      &= B \cap \left( \mathbb{R} \setminus \bigcup_{F \in \mathcal{A} \setminus \{B\}
      } (\mathbb{R}\setminus F) \right)  \\
      &= B \setminus \bigcup_{F \in \mathcal{A} \setminus \{B\}  } (\mathbb{R} \setminus
      F)
    .\end{align*}
    From this, we can see that \( (\mathbb{R} \setminus F)_{F \in \mathcal{A}\setminus
    \{B\}  } \) is a cover of \( B \). And since \( F \) is closed for all \( F
    \in A\), this is an open cover. Now, \( B \) is both closed and bounded,
    implying that it must be compact, and hence, there is a finite subcover \(
    \mathbb{R}\setminus F_{1}, \mathbb{R}\setminus F_{2}, \ldots , \mathbb{R}
    \setminus F_{n} \) of \( B \). From this, we can proceed similarly to the
    set manipulation above, but in reverse to get:
    \[
      B \cap  F_{1} \cap  F_{2} \cap  \ldots \cap  F_{n} = \varnothing
    .\] 
  \item This follows directly from problem \ref{2A1}.
    \begin{alignat*}{3}
      |(a, b)| &= |(a, b) \cup \{a, b\}  | &&= |[a, b]| &&= b - a\\
      |(a, b]| &= |(a, b) \cup \{b\}  | &&= |[a, b]| &&= b - a\\
      |[a, b)| &= |(a, b) \cup \{a\}  | &&= |[a, b]| &&= b - a
    .\end{alignat*}
    The sets \( \{a\} , \{b\}    \) and \( \{a, b\}   \), needless to say, has
    outer measure \( 0 \) due to countability.
  \item \label{2A7} WLOG assuming \( a \le c \).

    If \( c \in (a, b) \), then \( (a, b) \cup (c, d) = \left( a, \max \{b,
    d\}   \right)  \). The outer measure is therefore \( |(a, b) \cup (c, d)| =
    \max \{b, d\}  - a < (b + d - c) - a = (b - a) + (d - c) \).
    
    If \( c \ge b \), then \( (a, b) \cup (c, d) =\varnothing \). From
    subadditivity of outer measure, we have \( |(a, b) \cup (c, d)| \le (b - a)
    + (d - c)\). To prove equality, an open interval cover \( \mathcal{A} \) of \( (a,
    b) \cup  (c, d) \):
    \begin{align*}
      (a, b) \cup (c, d) &\subseteq \bigcup_{C \in \mathcal{A}} C
    .\end{align*}
    Then, we have two open interval covers of \( (a, b) \) and \( (c, d) \):
    \begin{align*}
      (a, b) &\subseteq \bigcup_{F \in \mathcal{A}} (F \cap (a, b))\\
      (c, d) &\subseteq \bigcup_{F \in \mathcal{A}} (F \cap (c, d))\\
    .\end{align*}
    Note that here, we used the fact that the intersection of two open intervals
    is always another open interval. Additionally, we have:
    \[
      \ell(C) \ge  \ell(C \cap (a, b)) + \ell(C \cap (c, d))
    ,\] implying:
    \[
      \sum_{F \in \mathcal{A}} \ell(F) \ge 
      \sum_{F \in \mathcal{A}} \ell(C\cap (a, b)) +
      \sum_{F \in \mathcal{A}} \ell(C\cap (c, d))
      \ge |(a, b)| + |(c, d)| = (b - a) + (d - c)
    .\] 
    Hence, we have \( |(a, b) \cup  (c, d)| \ge  (b - a) + (d - c) \). QED.

    If \( c \le a \), then either \( d \le a \) (easily proven similarly to the
    \( c \ge b \) case) or \( d > a \) (similarly to the \( c \in (a, b) \)
    case).
  \item \label{2A8} We proceed similarly to the second case of problem
    \ref{2A7}. However,
    there is a major difference: the fact that \( F \cap  (\mathbb{R}
    \setminus (-t, t)) \) is not necessarily an open interval. However, we can
    circumvent this problem by:
    \begin{align*}
      F \cap  (\mathbb{R} \setminus (-t, t)) = (F \cap (-\infty, -t]) \cup  (F \cap
      [t, +\infty))
    .\end{align*}
    RHS is the union of two (may be half-closed) intervals. So, we need to also
    tweak the definition for the outer measure:
    \[
      |A| = \inf \left\{ \sum_{n=1}^{\infty} \ell(I_{n}),
      I_{1..+\infty} \text{ is an interval cover of } A
    \right\}
    .\] 
    We can show that this definition is in fact equivalent to the original one.
    For every interval \( I \), there always exists some open interval \( o(I,
    \varepsilon) \) containing \( I \) such that \( \ell(o(I, \varepsilon)) <
    \ell(I) + \varepsilon \).  Then, suppose \( I_{1..+\infty} \) is an interval
    cover of \( A \), we can "cover this cover" by the open interval cover \(
    \ell(I_{k}, \varepsilon_{k})_{k \in 1..+\infty} \). We pick \(
    \varepsilon_{k} \) such that the series \( \sum_{k=1}^{\infty}
    \varepsilon_{k} \) converges to some \( \varepsilon_{0} \), then we have:
    \[
      \sum_{n = 1}^{+\infty} \ell(I_{n})
      >
      \sum_{n = 1}^{+\infty} \ell(o(I_{n}, \varepsilon_{n})) - \sum_{n =
      1}^{\infty} \varepsilon_{n} = \sum_{n = 1}^{+\infty} \ell(o(I_{n},
      \varepsilon_{n}))  - \varepsilon_{0} > |A| -\varepsilon_{0}
    ,\] here \( |A| \) denotes the original definition for the outer measure of
    \( A \).

    Letting \( \varepsilon_{0} \to 0 \), we have:
    \[
      |A|' = \inf \left\{ \sum_{n=1}^{\infty} \ell(I_{n}),
      I_{1..+\infty} \text{ is an interval cover of } A  \right\} \ge |A|
    .\] 
    Needless to say, \( |A| \le |A|' \) due to the condition of \( |A|' \)
    being less strict than \( |A| \). Hence, the two definitions are equivalent.

    For the sake of completeness, we will sketch a proof for the original
    problem. Let \( \mathcal{A} \) be an open interval cover of \( A \), then we
    can find two interval covers of \( A \cap (-t, t) \) and \( A \cap
    (\mathbb{R} \setminus (-t, t)\):
    \begin{align*}
      A \cap (-t, t) &\ge \bigcup_{F \in \mathcal{A}} (F \cap (-t, t))\\
      A \cap (\mathbb{R} \setminus (-t, t)) &= \bigcup_{F \in \mathcal{A}} \left(
      (F \cap  (-\infty, -t] \cup 
      (F \cap  (t, +\infty]
    \right) 
    ,\end{align*} satisfying:
    \[
      \ell(F) = \ell(F \cap (-t, t)) + \ell(F \cap (-\infty, -t)) + \ell(F \cap
      (t, +\infty)), \forall  F \in \mathcal{A}
    .\] 
    Hence, we end up with the inequality:
    \[
      |A| \ge |A \cap (-t, t)| + |A \cap (\mathbb{R} \setminus (-t, t))|
    ,\] and by subadditivity of the outer measure, we have equality.
  \item \label{2A9}
    Take any sequence \( t_{n} \) that converges to \( +\infty \) as \( n \to
    +\infty \). Denote \( A_{n} = A \cap (-t_{n}, t_{n}) \), \( B_{n} = A_{n} \setminus
    A_{n - 1}\), we have:
    \begin{align*}
      |A_{n}| &=  \left| \bigcup_{k = 1}^{n} B_{k} \right| \\
      &= \left| \bigcup_{k = 1}^{n} B_{k} \cap (-t_{n-1}, t_{n-1}) \right| 
      + \left| \bigcup_{k = 1}^{n} B_{k} \setminus (-t_{n-1}, t_{n-1}) \right|\\
      &= \left|  \bigcup_{k=1}^{n-1} B_{k} \right| + |B_{n}|\\
      &= |A_{n-1}| + |B_{n}|
    .\end{align*}
    From this recurrence relation, we have:
    \[
      |A_{n}| = \sum_{k = 1}^{n} |B_{k}|
    .\] 
    Letting \( n \to \infty \), we have:
    \[
      \lim_{n \to \infty} |A_{n}| = \sum_{k = 1}^{+\infty} |B_{k}|
    .\] 
    Now, note that \( (B_{k})_{k \ge 1} \) is a cover of \( A \), so by
    subadditivity, we have:
    \[
      \lim_{n \to \infty} |A_{n}| = \sum_{k \ge 1} |B_{k}| \ge |A|
    .\] 
    On the other hand, \( \lim_{n \to \infty} |A_{n}| \le |A| \) due to the fact
    that \( A_{n} \subseteq A \) for all \( n \in \mathbb{N} \).

    Hence, we have:
    \[
      \lim_{n \to \infty} |A \cap (-t_{n}, t_{n})| = |A|
    ,\] for all \( t_{n} \to +\infty \), or in other words:
    \[
      \lim_{t \to \infty} |A \cap (-t, t)| = |A|
    .\] 
\item
  We knew \( |\mathbb{Q}| = 0 \), so \( |\mathbb{Q} \cap [0, 1]| = 0 \).

  Letting \( A = Q \cap [0, 1] \) and \( B = Q \setminus [0, 1] \) in problem
  \ref{2A1}, we have QED.
\item 
  Consider an interval cover \( (C_{k})_{k \in \mathbb{N}^{*}} \) of \( S =
  \bigcup_{k=1}^{+\infty} I_{k}  \). Denote \( C^{m}_{k} = C_{k} \cap I_{m} \),
  then, by intuition, we must have:
  \[
    \sum_{m = 1}^{+\infty} \ell(C^{m}_{k}) = \ell(C_{k})
  .\] 
  Note that both \( C_{k} \) and \( C^{m}_{k} \) are open intervals, so this
  identity is totally unrelated to the outer measure. Proving this rigorously
  could be done by looking at the boundary points of the intervals \( C_{m} \)
  and \( I_{k} \).

  From this, we must have:
  \begin{align*}
    \sum_{k = 1}^{+\infty} \ell(C_{k}) &= \sum_{k = 1}^{+\infty} \sum_{m =
    1}^{+\infty} \ell(C^{m}_{k}) \\
    &= \sum_{m = 1}^{+\infty} \sum_{k = 1}^{+\infty} \ell(C^{m}_{k}) \\
    &\ge \sum_{m = 1}^{+\infty} |I_{m}| \text{, (since } C^{m \in 1..+\infty}_{k}
    \text{is an open interval cover of } I_{m} \text{)} \\
    &= \sum_{k = 1}^{+\infty} \ell(I_{m})
  .\end{align*}
  Here, switching the order of summation is justified by Tonelli's theorem.

  We have equality if \( C_{k} = I_{k}, \forall k \in \mathbb{N}^{*} \). Hence,
  infimum of the LHS, i.e. \( |S| \), must be:
  \[
    |S| = \sum_{k = 1}^{+\infty} \ell(I_{m})
  .\] 
\item \label{2A12}
    \begin{enumerate}[label=(\alph*)]
      \item We need to prove that the set:
        \[
          F^{c} = \bigcup_{k = 1}^{\infty} \left( r_{k} - \frac{1}{2^{k}}, r_{k}
          + \frac{1}{2^{k}}\right) 
        \] is an open set. This is trivial, since it is a (countable) union of
        open sets. For completeness sake, consider \( x_{0} \in F^{c} \), then
        there exists some \( k \) such that \( x_{0} \in \left( r_{k}-2^{-k},
        r_{k}+2^{-k} \right)  \), which is an open set. Hence, there exists some
        \( \varepsilon > 0 \) such that \( B(x_{0},\varepsilon) \subseteq
        (r_{k}-2^{-k}, r_{k}+2^{-k}) \subseteq F^{c} \). Therefore \( F^{c} \)
        is open and \( F \) is closed.
    \item Every non-degenerate interval (intervals with non-zero outer measure)
      could be shown to contain a rational number. This can be easily proven by
      consider an arbitrary interval \( I \) with \( |I| > 0 \) and a closed
      interval \( [a, b] \subseteq I \) such that \( a < b \).
      By the fact that \( \mathbb{Q} \) is a
      dense subset of \( \mathbb{R} \), there exists some \( q \in \mathbb{Q} \)
      such that \( q \in [a, b] \), and therefore \( q \in I \).

      Hence, if some interval \( I \) with non-zero outer measure is contained
      in \( F \), then \( \exists q \in \mathbb{Q} \) such that \( q \in F \).
      This contradicts with the fact that \( r_{1..+\infty} \) is an enumeration
      of \( \mathbb{Q} \).
    \item By subadditivity, we have:
      \[
        |F^{c}| \le \sum_{k =1}^{+\infty} \left| (r_{k}-2^{-k}, r_{k}+2^{-k})
        \right| = \sum_{k =1 }^{+\infty} 2^{1-k} = 2
      .\] 
      If \( F \) has finite outer measure, then:
      \( +\infty > |F| + |F^{c}| \ge |\mathbb{R}| \), which contradicts with \(
      |\mathbb{R}| = +\infty\). Hence, \( |F| = +\infty \).
    \end{enumerate}

  \item \( F \) could be taken as a restricted version of the \( F \) in problem
    \ref{2A12}:
    \[
      F = [0, 1] \setminus \bigcup_{k = 1}^{\infty} \left( r_{k} -
      \frac{\varepsilon}{N 2^{k}}, r_{k} + \frac{\varepsilon}{N 2^{k}} \right) 
    ,\] with \( r_{1..+\infty} \) is an enumeration of \( [0, 1] \cap \mathbb{Q}
    \).

    By the same logic, \( F \) is closed, contains no irrational numbers, and
    its complement has outer measure at most \( \frac{2\varepsilon}{N} \).

    Hence,
    \[
      |F| \ge |[0, 1]| - |F^{c}| \ge 1 - \frac{2\varepsilon}{N}
    .\] 

    To make \( 1 - \frac{2\varepsilon}{N} > 1 - \varepsilon \), one simply takes
    \( N = 3 \).
\item We will go straight to (f). The reason why the result differs is that the
  colored region is not a triangle:
  \[
    \frac{11}{5} \neq \frac{20}{9} \neq \frac{20 - 11}{9 - 5}
  .\] 
  There are plenty of videos that explain this "paradox" (The Infinite Chocolate
  Paradox), so we will only leave it at that.
\end{enumerate}

% section Exercises 2A (end)

\section{Exercises 2B} % (fold)
\label{sec:Exercises 2B}

\begin{enumerate}[label=\textbf{2B.\arabic*}]
  \item Let \( I(K) = \bigcup_{n e K} (n, n + 1] \). Then, we trivially have:
    \begin{itemize}
      \item \( \varnothing = I(\varnothing) \);
      \item \( I(K)^{c} = I(K^{c}) \);
      \item \( \bigcup_{k = 1}^{\infty} I(K_{k}) = I \left( \bigcup_{k =
        1}^{\infty} K_{k} \right)  \) for \( K_{1}, K_{2}, \ldots \subseteq
        \mathbb{Z} \).
    \end{itemize}
  \item 
    \begin{itemize}
    \item 
    Consider \( E \subseteq X \) such that \( E \) is countable. Then, there
    exists an enumeration \( x_{1}, x_{2}, \ldots  \) of \( E \), and we have:
    \[
      E = \bigcup_{k = 1}^{\infty} \{x_{k}\}
    .\] 
    Since \( \{x_{k}\} \in \mathcal{A}, \forall k \in \mathbb{N}^{*}  \), \( E
    \in \mathcal{A} \).

    On the other hand, \( \mathcal{A} \) is a \( \sigma \)-algebra, so \( X
    \setminus E \in \mathcal{A} \). Hence, \( \mathcal{A} \) contains all
    subsets \( E \) of \( X \) such that \( X \setminus E \) or \( E \) is
    countable.

    In the previous example, we proved that \( \mathcal{A}' \), the set of all
    \( E \) such that \( E \) or \( X \setminus E \) is countable, is a \(
    \sigma \)-algebra. Hence, by the minimality of \( \mathcal{A} \), we must
    have \( \mathcal{A} = \mathcal{A}' \).

  \item Trivial (but tedious).
    \end{itemize}

  \item \label{2B3}
    Consider \( a \in \mathbb{R} \), if we can prove that \( (a, +\infty) \in
    \mathcal{S} \), then \( \mathcal{S} \) is the collection of Borel subsets of
    \( \mathbb{R} \).

    Since \( a \in \mathbb{R} \), there exists an increasing sequence of
    rationals \( q_{1}, q_{2}, \ldots  \) that converges to \( a \). Then, we
    have:
    \[
      (a, +\infty) = \bigcap_{n = 1}^{\infty} (q_{n}, q_{n} + n]
    ,\] and therefore \( (a, +\infty) \in \mathcal{S} \).
  \item This trivially follows from:
    \[
      (r, s] = (r, n) \setminus (s, n)
    ,\] with \( r, s \in \mathbb{Q} \), \( n \) is an arbitrary integer that is
    greater than \( r \) and \( s \).
  \item In problem \ref{2B3}, we have:
    \[
      (a, +\infty) = \bigcap_{n = 1}^{\infty} (q_{n}, q_{n} + n] = \bigcap_{n =
      1} \bigcup_{k = 2}^{2n} \left( q_{n} + \frac{k}{2} - 1, q_{n} + \frac{k}{2}
      \right) \in \mathcal{S}
    .\]
  \item This is even easier than the previous problem:
    \[
      (a, +\infty) = \bigcap_{n = 1}^{\infty} [q_{n}, +\infty)
    .\]
  \item \label{2B7}
    Let \( \mathcal{B} \) be the \( \sigma \)-algebra of Borel subsets of \(
    \mathbb{R}\). Then, consider \( \mathcal{B}' = \{A \in \mathcal{B}, A +
    \{t\} \in \mathcal{B} \} \subseteq \mathcal{B}  \). We have:
    \[
      (a, +\infty) = (a - t, +\infty)  + \{t\} \subseteq \mathcal{B}' 
    ,\] so to prove \( \mathcal{B} = \mathcal{B}' \), we just need to prove that
    \( B'\) is a \( \sigma \)-algebra:
    \begin{itemize}
    \item \( \varnothing = \varnothing + \{t\} \in \mathcal{B}  \), so \(
      \varnothing \in B' \).
    \item If \( E \in \mathcal{B}' \), then \( \mathbb{R} \setminus E \in
      \mathcal{B} \) and \( (\mathbb{R} \setminus E) + \{t\} = \mathbb{R}
      \setminus (E + \{t\}  ) \subseteq \mathcal{B}  \). Therefore, \(
      \mathbb{R} \setminus E \in \mathcal{B}' \).
    \item If \( E_{1}, E_{2}, \ldots \in \mathcal{B}' \), then the countable
      union of \( E_{1}, E_{2}, \ldots  \) is in \( \mathcal{B}' \) and:
      \[
        \left( \bigcup_{n = 1}^{\infty} E_{n} \right) + \{t\}   = \bigcup_{n =
        1}^{\infty} \left( E_{n} + \{t\}   \right) \in \mathcal{B}'
      .\] Hence, \( \bigcup_{n = 1}^{\infty} E_{n} \in \mathcal{B}' \).
    \end{itemize}
  \item Proceed similarly as in problem \ref{2B7}.
  \item Consider any non-Borel set \( A \). Then,
    \[
      f(x) = \begin{cases}
        1, &\text{ if } x \in A\\
        -1, & \text{ otherwise }
      \end{cases}
    ,\] has \( |f|(x) = 1 \) for all \( x \in \mathbb{R} \), which is
    measurable.

    Meanwhile, \( f^{-1}((0, +\infty)) = A \), which is not a Borel set, so \( f
    \) is not a measurable function.
  \item 
    Let \( E \) be the set of real numbers that have a decimal expansion with
    the digit \( 5 \) appearing infinitely often. Denote \( E_{+} = E \cap [0,
    +\infty), E_{-} = E \cap (-\infty, 0]   \), then \( E_{-} = (-1)E_{+} \), so
    if \( E_{+} \) is a Borel set, \( E_{-} \) is also a Borel set and therefore
    \( E = E_{+} \cup E_{-} \) is a Borel set.

    Let \( E_{k} \) be the set of positive real numbers with the \( k \)-th digit
    after the decimal point being \( 5 \). Then, \( E_{k} \) could be written in
    the form:
    \[
      E_{k} =  \frac{1}{10^{k-1}}\bigcup_{n \in \mathbb{N}}\left[ n + 0.5, n +
      0.6 \right) = \frac{1}{10^{k-1}}E_{1} \in \mathcal{B}
    .\] 
    Then, \( \mathbb{R} \setminus E_{+} \), the set of all positive real numbers
    with finitely many \( 5 \) in their decimal representation, could be written
    as:
    \[
      \mathbb{R} \setminus E_{+} = \bigcup_{S \in F(\mathbb{N}^{*})} \left( 
        \bigcap_{k \in S} E_{k} \setminus \bigcup_{k \in S^{c}} E_{k}
      \right)
    ,\] with \( F(\mathbb{N}^{*}) \) denoting the set of all finite subsets of
    \( \mathbb{N} \), which is countable (such a set \( S = \{s_{1}, s_{2},
    \ldots, s_{k} \}   \) could be uniquely identified as
    the natural number \( p_{1}^{s_{1}}p_{2}^{s_{2}}\ldots p_{k}^{s_{k}} \),
    with \( p_{1}, p_{2}, \ldots  \) are the prime numbers).

    Hence, \( \mathbb{R} \setminus E_{+} \) is a Borel set, and therefore \( E_{+}
    \) and \( E \) are also a Borel sets.
  \item \begin{enumerate}[label=(\alph*)]
    \item Denote \( \mathcal{S}' = \{F \cap X: F \in \mathcal{T}\}   \). We will
      prove \( \mathcal{S} = \mathcal{S}' \) by proving \( \mathcal{S}
      \subseteq \mathcal{S}' \) and \( \mathcal{S}' \subseteq \mathcal{S}\).
      
      First, for every \( E \in \mathcal{S} \), we have \( E \in \mathcal{T} \)
      and \( E \subseteq X \). Hence, \( E = E \cap X \), and thus \( E \in
      \mathcal{S}' \).

      Secondly, for every \( F \in \mathcal{T} \), denote \( E = F \cap X \).
      Then, \( E \subseteq X \), so to prove \( E \in \mathcal{S} \), one only
      needs \( E \in \mathcal{T} \). This trivially follows from the fact that
      \( E = F \cap X \), with \( F \) and \( X \) are both in the \( \sigma
      \)-algebra \( \mathcal{T} \).
    \item This trivially follows from:
      \begin{itemize}
        \item \( X \subseteq X \) and \( X \in \mathcal{T} \) implies \( X \in
          \mathcal{S} \).
        \item If \( E \in \mathcal{S} \), then \( X \setminus E \in \mathcal{T}
          \) and \( X \setminus E \subseteq X \), therefore \( X \setminus E \in
          \mathcal{S}\).
        \item If \( E_{1}, E_{2}, \ldots  \in \mathcal{S} \), then \( E =
          \bigcup_{k = 1}^{\infty} E_{k} \) is both a subset of \( X \) and a
          set in \( \mathcal{T} \), hence \( E \in \mathcal{S} \).
    \end{itemize}
  \end{enumerate}
\item 
  \begin{enumerate}[label=(\alph*)]
    \item Let \( a \) be an arbitrary element of \( G_{k} \). We will prove that
      there exists some \( \varepsilon > 0 \) such that \( B(a, \varepsilon)
      \subseteq G_{k} \).

      First, since \( a \in G_{k} \), there exists some \( \delta \) such that
      \( \sup_{x, y \in B(a, \delta)} |f(x)-f(y)| < \frac{1}{k} \). This
      property also holds for subsets of \( B(a, \delta) \), i.e. for every \( S
      \subseteq B(a, \delta)\), we have: \( \sup_{x, y \in S} |f(x)-f(y)| <
      \frac{1}{k} \).

      Hence, if we pick \( \varepsilon < \delta \), for every \( a' \in B(a,
      \varepsilon) \) has some \( \delta' = \delta - \varepsilon \) such that \(
      B(a', \delta') \subseteq B(a, \delta)\) and therefore \( \sup_{x, y \in
      B(a', \delta')} |f(x)-f(y)| < \frac{1}{k} \). Hence, \( a' \in G_{k} \).

      Therefore, \( G_{k} \) contains a neighborhood of all \( a \in G_{k} \),
      making it an open set.
    \item \( f \) is continuous at \( x_{0} \) is equivalent to:
      \[
        \lim_{x \to x_{0}} f(x) = f(x_{0})
      ,\] or equivalently:
      \[
        \lim_{x \to x_{0}} |f(x)-f(x_{0})| = 0
      .\] 
      Using the \( \varepsilon-\delta \) definition of limits, this is
      equivalent to the fact that for every \( \varepsilon > 0 \), there exists
      some \( \delta > 0 \) such that \( \sup_{x \in B(x_{0}, \delta)}
      |f(x)-f(x_{0})| < \varepsilon \). By the triangle inequality, this
      implies:
      \[
        \sup_{x, y \in B(x_{0}, \delta)} |f(x)-f(y)| < 2\varepsilon
      .\] 
      Take \( \varepsilon = \frac{1}{2k} \), then the existence of \( \delta \)
      implies that \( x_{0} \in G_{k} \). Then, we must have \( x_{0} \in
      \bigcap_{k \in \mathbb{N}^{*}} G_{k} \).

      Now, if \( x_{0} \in G_{k} \) for all \( k \in \mathbb{N}^{*} \), then for
      every \( k \in \mathbb{N}^{*} \), there exists some \( \delta_{k} > 0 \) such
      that:
      \[
        \sup_{x, y \in B(x_{0}, \delta_{k})} |f(x)-f(y)| < \frac{1}{k}
      ,\] which implies:
      \[
        \sup_{x \in B(x_{0}, \delta_{k})} |f(x)-f(x_{0})| < \frac{1}{k}
      .\] 

      Hence, for every \( \varepsilon > 0 \), there always exists some \( k \in
      \mathbb{N}^{*}\)
      such that \( \frac{1}{k} < \varepsilon \) and \( \delta = \delta_{k} \)
      satisfies:
      \[
        \sup_{x \in B(x_{0}, \delta)} |f(x)-f(x_{0})|< \frac{1}{k} < \varepsilon
      .\] 
      Hence, \( f \) is continuous at \( x_{0} \).
    \item Since \( G_{k} \) are all open sets, they are also Borel sets. The
      intersection of such sets is also a Borel set. Hence, the set of points at
      which \( f \) is continuous, \( \bigcap_{k \in \mathbb{N}^{*}} G_{k} \),
      is a Borel set.
  \end{enumerate}

\item Since \( c_{1}, c_{2}, \ldots, c_{n}  \) are pairwise distinct, we must
  have:
  \[
    f^{-1}(\{c_{k}\}  ) = E_{k}
  .\] 
  Hence, if \( f \) is measurable, then \( E_{k} \in \mathcal{S} \) for all \( k
  \in 1..n\).

  On the other hand, if \( E_{k} \in \mathcal{S} \) for all \( k \in 1..n \),
  then:
  \[
    f^{-1}(A) = \bigcup_{c_{k} \in A} E_{k} \in \mathcal{S}
  ,\] for all \( A \subseteq \mathbb{R} \) (not necessarily a Borel set).
\item Let \( S \) be the complicated set:
  \[
    S = \bigcap_{n=1}^{\infty} \bigcup_{j=1}^{\infty} \bigcap_{k=j}^{\infty}
    (f_{j}-f_{k})^{-1}\left( \left( -\frac{1}{n}, \frac{1}{n} \right)  \right) 
  .\] 
  We have:
  \begin{align*}
    x \in S &\iff x \in \bigcup_{j=1}^{\infty} \bigcap_{k=j}^{\infty}
    (f_{j}-f_{k})^{-1}\left( \left( -\frac{1}{n}, \frac{1}{n} \right)  \right) ,
    \forall n \in \mathbb{N}^{*}\\
            &\iff \forall n \in \mathbb{N}^{*}, \exists j \in \mathbb{N}^{*}:
            \forall  k \ge j,
            |f_{j}-f_{k}|(x) <  \frac{1}{n}
  .\end{align*}
  The statement "\( \exists j \in \mathbb{N}^{*}: \forall k \ge j,
  |f_{j}-f_{k}|(x) < \frac{1}{n} \)" implies "\( \exists N \in \mathbb{N}^{*},
  \sup_{u, v \ge  N} |f_{u}-f_{v}|(x) < \frac{2}{n} \)", since with \( N = j \):
  \[
    \sup_{u, v \ge N} |f_{u}-f_{v}|(x) \le  \sup_{u\ge N}|f_{j}-f_{u}|(x) +
    \sup_{v\ge N}|f_{j}-f_{v}|(x) < \frac{2}{n}
  .\] 
  Hence, \( x \in S \) implies \( \forall n \in \mathbb{N}^{*}, \exists  N \in
  \mathbb{N}^{*}, \sup_{u, v \ge  N} |f_{u}-f_{v}|(x) < \frac{2}{n} \). Replacing
  \( n \) with \( \varepsilon \) yields the usual \( \varepsilon-\delta \)
  definition for Cauchy sequence, therefore \( f_{n}(x) \) is a Cauchy sequence.
  This is equivalent with the fact that \( f_{n}(x) \) converges.

  On the other hand, if \( f_{n}(x) \) converges, one can just work backwards
  like in the previous problem. From the statement "\( \exists N \in
  \mathbb{N}^{*}: \sup_{u,v \ge N} |f_{u}-f_{v}|(x) \)",
  one could only recover an weaker original statement: "\( \exists j \in
  \mathbb{N}^{*}: \forall k \ge j, |f_{j}-f_{k}|(x) < \frac{2}{n} \)". However,
  we have this equivalence:
  \[
            \forall n \in \mathbb{N}^{*}, \exists j \in \mathbb{N}^{*}:
            \forall  k \ge j,
            |f_{j}-f_{k}|(x) <  \frac{1}{n}
            \iff
            \forall n \in \mathbb{N}^{*}, \exists j \in \mathbb{N}^{*}:
            \forall  k \ge j,
            |f_{j}-f_{k}|(x) <  \frac{2}{n}
  ,\] so we still can deduce back \( x \in S \).

  From this, we can trivially see why \( S = \{x \in X: x_{1},x_{2}, \ldots
  \text{converges}\}   \) is a \( \mathcal{S} \)-measurable subset of \( X \).
\item \( \mathcal{S} \) is a \( \sigma \)-algebra due to the following:
  \begin{itemize}
  \item \( \varnothing = \bigcup_{k \in \varnothing} E_{k} \in \mathcal{S} \).
  \item \( X \setminus \bigcup_{k \in K} E_{k} = \bigcup_{k \in \mathbb{Z}^{+}
    \setminus K} E_{k} \in \mathcal{S} \).
  \item \( \bigcup_{n=1}^{\infty} \bigcup_{k\in K_{n}} E_{k} = \bigcup_{k \in K}
    E_{k} \in \mathcal{S}\), with \( K = \bigcup_{n =1}^{\infty} K_{n} \).
  \end{itemize}

  Consider \( f: X \to  \mathbb{R} \).
  \begin{itemize}
  \item If \( f \) is constant on \( E_{k} \) for all \( k \in \mathbb{Z}^{+}
    \), then \( f^{-1}(A) = \bigcup_{f(E_{k}) \subseteq A} E_{k} \in \mathcal{S}
    \). Hence, \( f \) is measurable.
  \item If \( f \) is measurable, then let \( e \in E_{n} \), we must have:
    \[
      f^{-1}(\{f(e)\} = \bigcup_{k \in K} E_{k}  
    ,\] since the LHS is in \( \mathcal{S} \). We trivially have \( n \in K \),
    so LHS must also include \( E_{k} \). Hence, \( f(x) = f(e), \forall x \in
    E_{k} \), and \( f \) is constant on \( E_{k} \) for all \( k \in
    \mathbb{Z}^{+} \).
  \end{itemize}

\item
  \begin{enumerate}[label=(\alph*)]
    \item Trivial
    \item If \( f \) is constant on \( A \), then \( f^{-1}(Y) \) is either \(
      \varnothing \in \mathcal{S}_{A} \) or \( X \in \mathcal{S}_{A} \). Hence,
      \( f \) is measurable wrt \( \mathcal{S}_{A} \).

      If \( f \) is measurable wrt \( \mathcal{S}_{A} \), then we will proceed
      as follows. If \( A \) is empty, then \( f \) is constant on \( A \) is
      vacuously true. Otherwise, take \( y \in f(A) \neq  \varnothing \), then
      the set \( f^{-1}(\{y\}  ) \in \mathcal{S}_{A} \) must have at least one
      element of \( A \), therefore we must have \( A \subseteq f^{-1}(\{y\}  )
      \), i.e. \( f(A) = \{y\}   \). Hence, \( f \) is constant on \( A \).
  \end{enumerate}

\item We prove that \( A = f^{-1}((a, +\infty)) \) is a Borel set. 
  Denote \(
  D\) as the countable set \( \{x \in X: f\text{ is discontinuous at } x\}   \).
  Then, if \( x \in X \setminus D \), then we can write:
  \[
    A = \underbrace{(A \cap D)}_{\text{countable}} \cup (A
    \setminus D)
  .\] 
  Now, we only need to prove that \( A\setminus D \) is a Borel set. We can
  proceed similarly to the proof of Theorem 2.41.

  If \( x_{0} \in A \setminus D \), then \( f \) is continuous at \( x_{0} \).
  Then, we have two cases:
  \begin{itemize}
  \item If \( f(x_{0}) \le  a \), then \( x_{0} \not\in A \setminus D \)
  \item Otherwise, if \( f(x_{0}) > a \), then there exists some neighborhood \(
    B(x_{0}, \varepsilon_{x_{0}}\) such that \( f(x) > a, \forall x \in B(x_{0},
    \varepsilon_{x_{0}}) \).
  \end{itemize}
  Hence, we can write \( A\setminus D \) as:
  \[
    A \setminus D = \bigcup_{x \in A} B(x_{0}, \varepsilon_{x_{0}})
  .\] 
  Note that this is a Borel set because it is open, not because it is a
  uncountable union of Borel sets.

  From here, it is trivial to conclude that \( A \) is a Borel set.
\item The derivative \( f' \) of \( f \) is simply the limit:
  \[
    f'(x) = \lim_{h \to 0} \frac{f(x+h)-f(h)}{h} = \lim_{n \to \infty} \frac{f
    \left( x +  \frac{1}{n} \right) - f(x) }{\frac{1}{n}}
  .\] 
  Here, we transform the original limit into the limit wrt \( n \) to have a
  sequential limit instead of an \( \varepsilon-\delta \) one. Define \(
  f_{n}(x) = n \left( f \left( x + \frac{1}{n} \right) - f(x) \right)  \), then
  \( f_{n} \to  f \) as \( n \to \infty \), and since all \( f_{n} \) are Borel
  measurable, \( f' \) must also be Borel measurable.

\item Let \( f: X \to \mathbb{R} \) be a \( \mathcal{S} \)-measurable function.
  First, we know \( f^{-1}((-\infty, a)), f^{-1}([a, +\infty)) \) is a partition
  of \( X \), so either set has to be countable. Denote \( I_{1}, I_{2} \) as
  follows:
  \begin{align*}
    I_{1} &= \{a \in f(X): f^{-1}((-\infty, a)) \text{ is countable}\}   \\
    I_{2} &= \{a \in f(X): f^{-1}([a, +\infty)) \text{ is countable}\}
  .\end{align*}
  Then, \( I_{1} \cap I_{2} = \varnothing \) and \( I_{1} \cup I_{2} = f(X) \).
  Now, take \( u \in I_{1}, v \in I_{2} \). If \( u > v \), then \(
  f^{-1}((-\infty, u)) \) and \( f^{-1}([v, +\infty) \) are countable, and
  hence:
  \[
    X = f^{-1}((-\infty, u)) \cup f^{-1}([u, +\infty))
    \subseteq f^{-1}((-\infty, u)) \cup f^{-1}([v, +\infty))
  \] is countable.

  Now, if \( X \) is countable, then any function \( f: X \to
  \mathbb{R}\) would be measurable, so that's the trivial case. Hence, we are
  only interested in the other case, when \( X \) is uncountable, which
  contradicts with the above statement.

  Hence, the \( u > v \) case could not happen, so \( u < v, \forall u \in
  I_{1}, v \in I_{2} \). This implies that there must be a point \( s \)
  separating the two sets, i.e.
  \[
    (-\infty, s) \subseteq I_{1} \text{ and } (s, +\infty) \subseteq I_{2}
  ,\] Here, \( s \) may be \( \pm \infty \), so either \( I_{1} \) or \( I_{2}
  \) may be empty. \( s \) could also be explicitly defined as \( s = \sup I_{1} =
  \inf I_{2} \). However, looking at these cases carefully, for example if \(
  s \) is \( +\infty \), then \( I_{1} = X \), so \( f^{-1}((-\infty, a) \) is
  countable for all \( a \in f(X) \). Then, we must have:
\( X = \bigcup_{n \in \mathbb{N}} f^{-1}((-\infty, n)) \) is countable, which
  contradicts with what we assumed above. Similarly, \( s \) could not be \(
  -\infty \) either, so \( s \in \mathbb{R} \).

  Now, consider the two sets:
  \begin{align*}
    S_{1} &= \bigcup_{n \in \mathbb{N}^{*}} f^{-1} \left( \left( -\infty, s -
    \frac{1}{n} \right)  \right) = f^{-1}((-\infty, s) ,\\
        S_{2} &= \bigcup_{n \in \mathbb{N}^{*}} f^{-1} \left( \left( s +
        \frac{1}{n}, +\infty \right)  \right)  = f^{-1}((s, +\infty)
  .\end{align*}
  Since they are countable union of countable sets, they must be countable.
  Hence, \( S_{1} \cup  S_{2} \) is also countable. This is equivalent with the
  fact that \( f^{-1}(\mathbb{R} \setminus \{s\}  ) \) is countable. Hence, \( f
  \) is constant except on countably many points.

  In the \( X \) is countable case, every function \( f \) is also constant
  except on countably many points. Hence, we have proven for all case, \( f \)
  is measurable on \( \mathcal{S} \) if and only if it is constant except on
  countably many points. We can trivially verify that \( f^{-1}(S) \) is either
  countable or has its complement countable for all \( S \subseteq \mathcal{S}
  \):
  \begin{align*}
    f^{-1}(S) &\subseteq \mathbb{R} \setminus f^{-1}(\{s\}  )  \text{ is
    countable, if } s \not\in S\\
    \mathbb{R} \setminus f^{-1}(S) &\subseteq \mathbb{R} \setminus f^{-1}(\{s\}  )  \text{ is countable, if } s \in S
  .\end{align*}
\item We have \( f(x)^{g(x)} = \exp (g(x) \ln f(x)), \forall  x \in X \). RHS is
  well-defined, as \( f(x) > 0 \) for all \( x \in X \). Now, we simply use
  Theorem 2.44 and 2.46 in MIRA to see that the function on RHS is measurable.
  The elementary functions \( \exp \) and \( \ln  \) are measurable as a result
  of their property of being continuous on their respective domains.
\item Trivial, as the set \( \{ (a, +\infty] \}   \) generates the \( \sigma
  \)-algebra of Borel subsets of \( [-\infty, +\infty] \).
\item 
  If one takes \( B = \mathbb{R} \setminus \mathbb{Q} \), then if \( f(x) = x,
  \forall x \in B \), which is obviously increasing, then \( f \) is not
  continuous at every \( x_{0} \in B \), since \( x_{0} \not\in
  \operatorname{Int} B \).

  Hence, the problem statement is somewhat ill-formed, and we might modify it to
  be: "proof that \( f \) is continuous at every element of \(
  \operatorname{Int} B \) except for a countable subset of \( \operatorname{Int}
  B\)". With this slight modification, we may assume \( B \) is open (consider
  the restriction of \( f \) to \( \operatorname{Int} B \)).

  Denote the set of discontinuities of \( f \) as \( D \subseteq B \).
  Since \( f \) is increasing, the one-sided limits of \( f \) at every point \(
  x_{0} \in B\) exist:
  \begin{align*}
    f_{+}(x_{0}) &= \lim_{x \to x_{0}, x > x_{0}} f(x) = \inf_{x \in B, x >
    x_{0}} f(x) \ge f(x_{0})\\
    f_{-}(x_{0}) &= \lim_{x \to x_{0}, x < x_{0}} f(x) = \sup_{x \in B, x <
    x_{0}} f(x) \le f(x_{0})
  .\end{align*}
  The two limits are equal if and only if \( f \) is continuous at \( x_{0} \).
  Hence, for all \( x_{0} \in D \), we must have \( f_{+}(x_{0}) > f_{-}(x_{0})
  \).

  Consider \( x_{1}, x_{2} \in D \), \( x_{1} \neq x_{2} \). We will prove that
  \( I(x_{1}) \) and \( I(x_{2}) \) does not have an intersection, with \(
  I(x_{0}) = (f_{-}(x_{0}), f_{+}(x_{0})) \). Assuming \( x_{1} < x_{2} \), then
  we must have:
  \[
    f_{+}(x_{2}) > f_{-}(x_{2}) \ge f(x_{1} + \varepsilon) \ge f_{+}(x_{1}) >
    f_{-}(x_{1})
  ,\] with \( \varepsilon \) being some small positive number such that \( x_{2}
  > x_{1} + \varepsilon\) and \( B(x_{1}, \varepsilon) \subseteq B \). Such \(
  \varepsilon \) always exists since \( B \) is open.

  Hence, every discontinuity \( x \) of \( f \) corresponds to an unique
  interval \( I(x) \). Since \( I(x) \) is nonempty, there exists some rational
  \( q(x) \in I(x) \). Since the intervals \( I(x) \) do not intersect, \( q(x)
  \) is unique for every \( x \in D \). Hence, \( q \) is an injection from \( D
  \) to \( \mathbb{Q} \), which suggests that \( |D| \le |\mathbb{Q}| \). Hence,
  \( D \) must be countable.
  
\item 
  First, we can trivially see that \( g = f^{-1} \) must be an increasing function.
  Hence, \( f^{-1} \) have one-sided limits.

  For every \( y_{0} \in f(\mathbb{R}) \), there exists \( x_{0} = f^{-1}(y_{0})
  \). Then, to prove \( f^{-1} \) is continuous at \( y_{0} \), we need to prove
  that:
  \[
    f^{-1}_{+}(y_{0}) = f^{-1}_{-}(y_{0}) = f^{-1}(y_{0}) = x_{0}
  .\] 
  Unpacking this, first considering \( f^{-1}_{+}(y_{0}) \), we need to prove:
  \[
    x_{0} = \inf_{y \in f(\mathbb{R}), y > y_{0}} f^{-1}(y) = \inf_{x \in
    \mathbb{R}, f(x) > y_{0}} x
  ,\] or \( x_{0} \) is the largest lower bound of the set \( \{x \in
  \mathbb{R}: f(x) > y_{0}\}   \). It is trivially a lower bound, so assuming
  the lower bound is \( L > x_{0} \). Now,
  consider some \( x_{1} \in (x_{0}, L) \), We have \( f(x_{1}) > f(x_{0}) >
  y_{0} \), so \( x_{1} \) is in the aforementioned set, but it is smaller than
  \( L \), which is a contradiction. Hence, \( L = x_{0} \), and therefore \(
  f^{-1}_{+}(y_{0}) =x_{0} \). Similarly, we have \( f^{-1}_{-}(y_{0}) = x_{0}
  \), and therefore \( f^{-1} \) is continuous at \( y_{0} \) for every \( y_{0}
  \in f(\mathbb{R})\).

\item 
  Denote \( \mathcal{B} \) as the \( \sigma \)-algebra containing all Borel
  subsets of \( \mathbb{R} \). Then, let \( \mathcal{S} = \{B \in \mathcal{B}:
  f(B) \in \mathcal{B}\}   \). We will prove that \( \mathcal{S} = \mathcal{B}
  \), by proving that \( \mathcal{S} \) is a \( \sigma \)-algebra containing all
  open intervals.

  \begin{itemize}
  \item \( \varnothing \in \mathcal{S} \), since \( \varnothing = f(\varnothing)
    \).
  \item If \( A \in \mathcal{S} \), then since \( f \) is injective, we must
    have \( f(\mathbb{R} \setminus A) = f(\mathbb{R}) \setminus f(A) \). Since
    \( \mathbb{R} \) is an open interval (\( (-\infty, \infty) \)), we can defer
    the proof of \( f(\mathbb{R}) \) until we prove all open intervals of \(
    \mathbb{R} \) is in \( \mathcal{S} \).
  \item If \( A_{1}, A_{2}, \ldots \in \mathcal{S} \), then \( \bigcup_{k =
    1}^{\infty} f(A_{k}) = f \left( \bigcup_{k = 1}^{\infty} A_{k} \right) \in
    \mathcal{B}  \), so \( \bigcup_{k = 1}^{\infty} A_{k} \in \mathcal{S} \).
  \end{itemize}

  Finally, we need to prove that \( \mathcal{S} \) contains all open intervals (note
  that \( \mathcal{S} \) being a \( \sigma \)-algebra depends on this claim, so
  we can't use the fact that \( \mathcal{S} \) is a \( \sigma \)-algebra here).

  Consider an open subset \( I \) of \( \mathbb{R} \) (yes, this argument not
  only works for open intervals, but also arbitrary open sets as well).
  First, consider the equivalence relation \( \sim  \)
  defined as:
  \[
    x \sim y \iff [x, y] \in \mathbb{R} \setminus f(I)
  .\] 
  This equivalence relation forms a partition of \( \mathbb{R} \setminus f(I)
  \). Consider an equivalence class \( C \), with \( u = \inf C, v = \sup C \),
  then for every \( \varepsilon > 0 \), \( u +\varepsilon, v - \varepsilon \in C
  \), and \( [u+\varepsilon, v - \varepsilon] \subseteq C \). Letting \(
  \varepsilon \to 0 \), we see that \( (u, v) \subseteq C \). Hence, \( C \)
  must be an (open, closed, half-open or half-closed) interval with endpoints \(
  u \) and \( v \).
  
  Consider the degenerate case when \( C \) is a singleton (\( u = v \)), \( C =
  \{u\}  \).

  For every \( \varepsilon > 0 \), consider the two sets \( B = (u - \varepsilon,
  u] \) and \( f(I) \). If the intersection \( B \cap f(I) \) is empty, then \(
  B \subseteq \mathbb{R} \setminus f(I)\). Then \( B \) must be a subset of the
  equivalence class of \( \sim  \) containing \( u \), or in other words \( B =
  (u - \varepsilon, u] \subseteq C = \{u\}  \), which is a contradiction.

  Hence, for every \( \varepsilon > 0 \), the intersection of \( (u -
  \varepsilon, u] \) with \( f(I) \) is non-empty, so one can always construct a
  increasing sequence \( x_{n} \in f(I) \) that converges to \( u \). Similarly,
  there exists a decreasing sequence \( y_{n} \in f(I) \) that also converges to
  \( u \):
  \[
    x_{1} < x_{2} < \ldots < u < \ldots < y_{2} < y_{1}
  .\] 
  Then, we must have:
  \[
    f^{-1}(x_{1}) < f^{-1}(x_{2}) < \ldots  < f^{-1}(y_{2}) < f^{-1}(y_{1})
  .\] 
  Note that here \( f^{-1}(x_{n}) \) and \( f^{-1}(y_{n}) \) exists due to \(
  x_{n}, y_{n} \in f(I) \). \( f^{-1}(u) \) does not exist so we did not
  include it here. However, this showed that the sequences \( f^{-1}(x_{n}) \)
  and \( f^{-1}(y_{n}) \) are monotonic and bounded, therefore converges:
  \begin{align*}
    a &= \lim_{n \to \infty} f^{-1}(x_{n})\\
    b &= \lim_{n \to \infty} f^{-1}(y_{n})
  .\end{align*}
  Now, note that \( y = \sup_{n \in \mathbb{N}^{*}} f(x_{n}) \le \sup_{n \in
  \mathbb{N}^{*}} f(a) = f(a) \), due to \( a \ge  x_{n} \) for all \(  n \in
  \mathbb{N}^{*} \), and similarly \( y \ge f(b) \). On the other hand, we must
  have \( a \le b \) since \( f^{-1}(x_{m}) < f^{-1}(y_{n}), \forall m, n \in
  \mathbb{N}^{*} \). Hence:
  \[
    f(a) \ge y \ge f(b) \ge f(a)
  .\] 
  And therefore \( a = b = f^{-1}(y) \). Note that \( a, b \in I \) since \( I
  \) is open (this is the only place that the openness of \( I \) is used),
  which contradicts with \( y \in \mathbb{R} \setminus f(I) \).

\item
  Simply define \( f_{k} \) as the linear interpolation between \( f \) and a
  strictly increasing function:
  \[
    f_{k}(x) = \frac{1}{2k}x + \left( 1 - \frac{1}{2k} \right) f(x)
  ,\] for all \(  k\in \mathbb{N}^{*} \).
\item The idea to construct \( g \) is as follows. For any \( x \notin B \), we
  find \( x_{1} \) and \( x_{2} \) in \( B \) which are closest to \( x \) from
  above or below, i.e.
  \begin{align*}
    x_{1} &= \sup \left( B \cap (-\infty, x) \right)  \\
    x_{2} &= \inf (B \cap (x, +\infty))
  .\end{align*}
  If \( x_{1}, x_{2} \) are both real, then we can define \( g(x) \) as the
  linear interpolation of \( f_{+}(x_{1}) \) and \( f_{-}(x_{2}) \):
  \[
    g(x) = \frac{x - x_{1}}{x_{2} - x_{1}}(f_{-}(x_{2}) - f_{+}(x_{1})) + f_{+}(x_{1})
  .\] 
  In the case \( x = x_{1} = x_{2} \), we can let the result of the division \(
  \frac{x - x_{1}}{x_{2} - x_{1}} = \frac{0}{0}\) to be any real number in \(
  (0, 1) \).

  To resolve the \( x_{1}, x_{2} = \pm  \infty\) case, consider a strictly increasing
  function \( h: \mathbb{R} \cup \{\pm \infty\} \to \mathbb{R}  \). For example,
  \( h = \arctan  \) is such a function, but any other \( h \)
  could work as well. Replacing the weight \( \frac{x - x_{1}}{x_{2}-x_{1}}
  \) as \( \frac{h(x) - h(x_{1})}{h(x_{2}) - h(x_{1})} \), and we are done:
  \[
    g(x) = \frac{h(x) - h(x_{1})}{h(x_{2}) -
    h(x_{1})}(f_{-}(x_{2})-f_{+}(x_{1})) + f_{+}(x_{1})
  .\] 
  The values of \( f_{+}(-\infty) \) and \( f_{-}(+\infty) \) could be taken as
  \( -M \) and \( M \), with \( M \) being an arbitrary bounding value of \( f \):
  \[
    f(x) < M, \forall x \in \mathbb{R}
  .\] 

\item Simply let:
  \[
    f(x) = \begin{cases}
      +\infty, &\text{ if } x \in B\\
      -\infty, &\text{ otherwise}
    \end{cases}
  \] for any non-Borel set \( B \), and we are done.
\item This trivially follows from:
  \[
    g^{-1}(I) = \begin{cases}
      f^{-1}(I) \cup (R \setminus B), &\text{ if } 0 \in I\\
      f^{-1}(I), &\text{ otherwise}
    \end{cases}
  .\] 
\item Simply take:
  \[
    f_{t}(x) = \begin{cases}
      1, &\text{ if } x = t \text{ and } t \in B\\
      0, &\text{ otherwise}
    \end{cases}
  ,\] and then:
  \[
    f(x) = \begin{cases}
      1, &\text{ if } x \in B\\
      0, &\text{ otherwise}
    \end{cases}
  \] is trivially non-measurable for any non-Borel set \( B \).
\item If \( x \) is rational, then for all large \( j \), we must have \( 2 \mid
  j!x \in \mathbb{Z}\), then \( \cos (j!\pi x) = 1 \) and therefore the limit is
  \( 1 \):
  \[
    \lim_{j \to \infty} \lim_{k \to \infty} (\cos (j!\pi x))^{2k} = \lim_{j \to
    \infty} \lim_{k \to \infty} 1^{2k} = 1
  .\] 
  If \( x \) is irrational, then \( j!x \) is non-integral for all \( j \in
  \mathbb{Z} \). Hence, \( -1 < \cos (j!\pi x) < 1 \) for all \( j \in
  \mathbb{Z} \), and therefore the inner limit \( \lim_{k \to \infty} (\cos
  (j!\pi x))^{2k} \) is zero. Then, we have:
  \[
    \lim_{j \to \infty} \lim_{k \to \infty} (\cos (j!\pi x))^{2k} = \lim_{j \to
    \infty} 0 = 0
  .\] 
\end{enumerate}

% section Exercises 2B (end)

\section{Exercises 2C} % (fold)
\label{sec:Exercises 2C}


\begin{enumerate}[label=\textbf{2C.\arabic*}]
  \item \( \mu (S) \ge \mu (E) \) for all \( E \subseteq S \), so \( S = \max
    \{\mu (E): E \subseteq S\} = \max [0, 1)  \), which is undefined. Hence,
    contradiction.
  \item \( w_{k} \) is trivially defined as \( w_{k} = \mu (\{k\}  ) \).
  \item \label{2C3} Simply define \( \mu \) such that \( \mu (k) = 2^{-k} \), then we
    must have \( \mu (E) \le \mu (\mathbb{Z}^{+}) = \sum_{k = 1}^{\infty} 2^{-k}
    = 1\). To prove that \( \mu (E) \) can take any value \( r \in [0, 1] \) is
    more trickier. First, the case \( r = 1 \) is done, so we would be only
    interested in the case \( r \in [0, 1) \). Then, take the \( r \) binary
    expansion of \( r \) and we are done:
    \[
      r = \sum_{k \in \mathbb{Z}} 2^{k}b(r, k) = \mu (\{k: b(r, -k) = 1\}  )  
    ,\]
  \item Consider the \( \sigma \)-algebra \( \mathcal{S} \) containing all sets
    \( E(K, B) = \left(   \bigcup_{k \in K} (-3(k + 1), -3k) \right) \cup B \),
    with \( K \subseteq \mathbb{N} \), \( B \) is a Borel subset of \( (0, 1)
    \). 

    We first start with proving \( \mathcal{S} \) is a \( \sigma \)-algebra:
    \begin{itemize}
    \item \( \varnothing = E(\varnothing, \varnothing) \).
    \item Take \( X = E(\mathbb{N}, (0, 1)) \), then \( X \setminus E(K, B) =
      E(\mathbb{N} \setminus K, (0, 1) \setminus B) \in \mathcal{S}\).
    \item The union of \( E(K_{1}, B_{1}), E(K_{2}, B_{2}), \ldots  \) is the
      set:
      \[
        \bigcup_{k = 1}^{\infty} E(K_{k}, B_{k}) = E \left( \bigcup_{k =
      1}^{\infty} K_{k}, \bigcup_{k = 1}^{\infty} B_{k} \right)
      .\]
    \end{itemize}

    We can let \( \mu  \) be the outer measure, but for a more simpler and more
    straightforward approach, we define:
    \[
      \mu (E(K, B)) = 3n(K) + |B|
    ,\] with \( n(K) \) denoting the cardinality of \( K \)\footnote{Commonly
    written as \( |B| \), but that notation was already used for the outer
  measure}.

  Then, it is trivial to verify that \( \mu  \) is a measure:
  \begin{itemize}
  \item \( \mu (\varnothing) = \mu (E(\varnothing, \varnothing)) =
    3n(\varnothing) + |\varnothing| = 0 \)
  \item If \( E(K_{1}, B_{1}), E(K_{2}, B_{2}), \ldots  \in \mathcal{S} \),
    are pairwise disjoint, then \( K_{1}, K_{2}, \ldots  \) and \( B_{1}, B_{2},
    \ldots \) are also two pairwise disjoint sequences of sets.
    Then:
    \begin{align*}
    \mu \left( \bigcup_{k = 1}^{\infty} E(K_{k}, B_{k}) \right) 
    &= \mu \left( E \left( \bigcup_{k = 1}^{\infty} K_{k}, \bigcup_{k =
    1}^{\infty} B_{k} \right)  \right)\\
    &= 3n \left( \bigcup_{k = 1}^{\infty} K_{k} \right) + \left| \bigcup_{k =
    1}^{\infty} B_{k} \right|   \\
    &= 3\sum_{k = 1}^{\infty} n(K_{k}) + \bigcup_{k = 1}^{\infty} |B_{k}|\\
    &= \sum_{k = 1}^{\infty} \mu (E(K_{k}, B_{k}))
    .\end{align*}
  \end{itemize}

  To conclude, we just need to show the following:
  \begin{itemize}
  \item \( \mu (E(K, B)) = 3n(K) + |B| \in \{\infty\} \cup \bigcup_{k =
    0}^{\infty} [3k, 3k + 1]  \) for all \( K \subseteq \mathbb{N} \) and Borel
    \( B \subseteq (0, 1) \).
  \item \( \mu (E(\mathbb{N}, (0, 1))) = \infty \).
  \item For \( r \in [3k, 3k + 1] \) for some \( k \in \mathbb{N} \), just take
    \( K = \{1, 2, 3, \ldots, k\}   \) and \( B = (0, r - 3k) \), then \( \mu
    (E(K, B)) = 3k + r - 3k = r \).
  \end{itemize}
  
  So basically, we have shown that \( \{\mu (E): E \in \mathcal{S}\} =
  \{\infty\} \cup \bigcup_{k = 1}^{\infty} [3k, 3k + 1]    \).

\item We have:
  \[
    \infty > \mu (X) \ge \mu \left( \bigcup_{A \in \mathcal{A}} A \right) =
    \sum_{A \in \mathcal{A}} \mu (A)
  .\] 
  Define \( S(\varepsilon) = \{A \in \mathcal{A}: \mu (A) > \varepsilon\}   \)
  Then, we have:
  \[
    \mu (X) \ge \sum_{A \in \mathcal{A}} \mu (A) \ge 
    \ge \sum_{A \in S(\varepsilon)} \mu (A) \ge \varepsilon n(S(\varepsilon))
  ,\] for all \( \varepsilon > 0 \).
  Hence, \( S(\varepsilon) \le \frac{\mu (X)}{\varepsilon} < \infty \). Hence,
  \( S(\varepsilon) \) are finite for all \( \varepsilon > 0 \).

  Therefore,
  \[
    \mathcal{A} = \bigcup_{n = 1}^{\infty} S \left( \frac{1}{n} \right) 
  ,\] must be countable.

\item We have \( c = \mu (S) < \infty \), and:
  \[
    \{\mu (E): E \in \mathcal{S}\} = \{\mu (S \setminus E): E \in \mathcal{S}\}
    = c - \{\mu (E): E \in \mathcal{S}\}  
  ,\] so \( S \) must be "symmetric". Then, it is trivial to show that \( c = 4
  \) is the only value that satisfies this condition.

  To construct \( (X, \mathcal{S}, \mu ) \), we can let:
  \begin{itemize}
    \item X = \( (0, 4) \)
    \item \( \mathcal{S} = \{B, B \cup \{\infty\}: B \text{ is a Borel set of }
      (0, 1)\}  \)
    \item \( \mu (B) = |B|, \mu (B \cup \{\infty\}) = 3 + |B|   \)
  \end{itemize}
  Then, this forms a measure satisfying \( \{\mu (E): E \in \mathcal{S}\} = [0,
  1] \cup [3, 4]  \). We will not get into details why, but one can
  trivially verify the conditions.
\item From
  \footnote{\href{https://math.stackexchange.com/q/4122427}{https://math.stackexchange.com/q/4122427}}:
  See problem
  \ref{2C12}. Here, we let \( X = \mathbb{R} \) and:
  \[
    f(x) = \begin{cases}
      x, &\text{ if } x \ge 3\\
      2^{-x}, &\text{ if } x \in \mathbb{Z}_{-}\\
      0, &\text{ otherwise}
    \end{cases}
  .\] 
  Then using the binary expansion trick in problem \ref{2C3}, for all \( r \in
  [0, 1] \), we can construct \( E \) such that \( \mu (E) = r \).

  If \( r \in [3, \infty]   \), simply take \( E = \{r\}   \) (if \( r \) is
  finite) or \( E = \mathbb{R} \) (if \( r = \infty \)).

  We can also easily show that \( \mu (E) \in [0, 1] \cup [3, +\infty] \) for all \(
  E \in \mathcal{S}\). That conclude our example.
\item From
  \footnote{\href{https://math.stackexchange.com/a/90820}{https://math.stackexchange.com/a/90820}}:
  Consider the \(
  \sigma \)-algebra of \( X = \{a, b, c\}, \mathcal{S} = P(X)  \), and a measur
  \( \mu  \) such that:
  \[
    \mu (\{a, b, c\}  ) = \mu (\{a, b\}  ) = \mu(\{b, c\}  ) = 1
  .\] with \( \mathcal{A} = \{\{a, b\}, \{a, c\}      \}   \). Then, either:
  \begin{itemize}
  \item \( \mu (\{a\}) = \mu(\{c\}) = 1, \mu(\{b\}  ) = 0   \), or
  \item \( \mu (\{a\}  )=\mu (\{c\}  ) = 0, \mu (\{b\}  ) = 1   \),
  \end{itemize}
  or neither of those cases. Picking \( \mu  \) and \( \nu  \) as the two
  measures defined by the two cases, we have the desired example.
\item Trivial.
\item Take \( E_{k} = (-\infty, -k) \cup (k, +\infty) \), then:
  \[ 
    \mu (E_{k}) = \infty \implies
    \lim_{k \to \infty} \mu (E_{k}) = \infty
  ,\] but:
  \[
    \mu \left( \lim_{k \to \infty} E_{k} \right)  = \mu (\varnothing) = 0
  .\] 
\item Trivial:
  \[
    \mu (C \cup D \cup E) = \mu (C) + \mu (D) + \mu (E) - \mu (C \cap D) - \mu
    (D \cap E) - \mu (C \cap E) + \mu (C \cap D \cap E)
  .\] 
\item \label{2C12} All measures of \( (X, \mathcal{S}) \) could be uniquely
  identified as a mapping \( f: X \to  [0, \infty]\):
  \[
    \mu (A) = \begin{cases}
      L - \mu (X \setminus A), &\text{ if } A \text{ is uncountable}\\
      \sum_{x \in X} f(x), & \text{ otherwise}
    \end{cases}
  ,\] 
  with \( L \) being an arbitrary value such that:
  \[
    L \ge \sum_{x \in X} f(x)
  .\] 
  (Equality is forced in the case \( X \) is countable.)
\end{enumerate}

% section Exercises 2C (end)

\section{Exercises 2D} % (fold)
\label{sec:Exercises 2D}

\begin{enumerate}[label=\textbf{2D.\arabic*}]
  \item By similar reasoning as in the Cantor set section in MIRA, we have the
    following lemma:

    \textbf{Lemma.} The set of all real numbers in \( [0, 1] \) not containing
    the digit \( 0 < d < n \) in its base \( n \) representation is a Borel set
    with Lebesgue measure 0. Equivalently, the set of all real numbers in \( [0, 1] \)
    containing \( d \) in its base \( n \) representation is also a Borel set,
    but with Lebesgue measure 1.

    Here, we will consider the case when \( d = 444\ldots 4 \) and \( n =
    10^{100} \).
    This is basically decimal, but the decimal digits are grouped in groups of
    100 digits. Despite not every 100 consecutive 4s belong to the same group
    and forming the digit \( d \), but if the digit \( d \) appears in the base
    \( n \) representation of some number \( r \in [0, 1] \), that number must
    contains 100 consecutive 4s in decimal representation.

    The set of such \( r \), by the above lemma, already has Lebesgue measure 1,
    so its superset, the set containing all numbers in \( (0, 1) \) containing
    100 consecutive 4s, must also has Lebesgue measure 1.

    This set is a Borel set, since it is the union of all \( B_{n} \), \( n \in
    \mathbb{N}\), with \( B_{n} \) denoting the set of real numbers in \(
    (0, 1) \) with the \( (n+1), (n+2), \ldots, (n + 100) \)-th digits being \(
    4 \)s. The sets \( B_{n} \) are union of finitely many closed intervals, and
    therefore Borel.

    \item If \( A \) is a Lebesgue measurable set, then there exists closed \( F
      \) such that \( |A  \setminus F| < \varepsilon = 1 \), then \( |F| \ge |A|
      - 1\), which contradicts with what we wanted. Hence, \( A \) must be a
      Lebesgue non-measurable set.

      Take the set from Theorem 2.18 in MIRA, which is called a
      \textbf{Vitali set}. To recap, a Vitali set \( V \subseteq (0, 1) \) is a
      set such that for every \( r \in \mathbb{R} \), there exists only one
      element \( v \in V \)
      such that \( r - v \in \mathbb{Q} \). The construction of \( V \) depends
      on the Axiom of Choice.

      Then, we have:
      \[
        \bigcup_{q \in [-1, 1] \cap \mathbb{Q}} \left( q  + V  \right) \subseteq
        [-2, 2]
      ,\] with the union on the LHS is an union of disjoint sets.

      Now, if \( F \) is a Borel set, and \( F \subseteq V \), then:
      \[
        \bigcup_{q \in [-1, 1] \cap \mathbb{Q}} \left( q  + F  \right) \subseteq
        \bigcup_{q \in [-1, 1] \cap \mathbb{Q}} \left( q  + V  \right) \subseteq
        [-2, 2]
      .\] 
      Note that here, the union on the LHS is a disjoint union of Borel sets, so
      we must have:
      \[
        \sum_{q \in [-1, 1] \cap \mathbb{Q}} |q + F| \le |[-2, 2]| = 4
      .\] 
      From this, and the fact that \( |q + F| = |F| \) for all \( q  \in
      \mathbb{R} \), proved that \( |F| = 0 \).

      Meanwhile, \( |V| > 0 \), since it's not a Lebesgue measurable set, so we
      must have:
      \[
        |V \setminus F| = |V|  - |F| = |V|, \forall \text{closed } F \subseteq
        V
      .\]
      Adding a scaling factor \( s > \frac{1}{|V|} \), we have:
      \[
        |sV \setminus F| = s \left| V \setminus \frac{1}{s}F \right|  = s|V| >
        1, \forall \text{closed } F \subseteq sV
      .\] 
      From this, we can simply take \( A = sV \).
\item Using the idea from the previous problem, first, we will consider the set
  \( V^{c} = [0, 1] \setminus V \). For every open set \( G \) containing \(
  V^{c}\), we must have:
  \[
    |G \setminus V| \ge |(G \cap [0, 1]) \setminus V| = |V^{c} \setminus 
    \underbrace{
    ([0, 1] \setminus G)
    }_{\text{closed}}| = |V^{c}| > 0
  .\] 
  Here, the outer measure of \( V^{c} \) is positive, since if it is zero, then
  \( V^{c} \) and therefore \( V \) are both Lebesgue measurable.

  To make this infinity, consider the set \( A \) defined as:
  \[
    A = \bigcup_{k = 1}^{\infty} (2k + V^{c})
  .\] 
  \( A \) is constructed from translating \( V^{c} \) so that they are
  completely separated. Intuitively, if we accept the following lemma:

  \textbf{Lemma.} If \( A \) and \( B \) are separated, i.e.
  \[
    \inf_{a \in A, b \in B} |b - a| > 0
  ,\] 
  then \( |A \cup B| = |A| + |B| \).

  We have, for every open \( G \) containing \( A \),
  \[
    |G \setminus A| \ge  \sum_{k = 1}^{\infty} |(G \setminus A) \cap [2k, 2k +
    1]|
    \ge \sum_{k = 1}^{\infty} |V^{c}| = \infty
  .\] 
The lemma could be proven easily by looking at an arbitrary open interval cover
of \( A \cup B \). We can avoid proving this by repeatedly using the result from
problem \ref{2A8} with \( t = \frac{3}{2}, \frac{5}{2}, \ldots, k + \frac{1}{2},
\ldots \).

\item \label{2D4}
  Once we have (a), the rest is easy.
  Since \( V \) is a countable union
  of intervals, it must be a Borel set. And since a set \( X \) can be written
  as:
  \[
    X = \bigcup_{x \in X} [x, x]
  ,\] one can simply takes \( X \) be any non-Borel set for the example of (c).

  Now, we will focus on (a). First, we will do the easier case for open
  intervals.
  Let \( F \) be a family of non-trivial open intervals.

  Denote the union of all intervals in \( F \) as \( V \), then for every \( x
  \in V \), there exists some open interval \( I(x) \in F \) containing \( x \).
  From the definition of open sets, there exists an open ball \( B(x,
  \varepsilon) \subseteq F \), but inside that open ball, there exists a
  \textbf{rational}\footnote{An open ball with rational center and rational
  radius. In the 1-dimensional case, it is an open interval with both endpoints
being rational numbers. Since such an open ball can be described from rational
numbers, there must be only an countable amount of them.} open ball \( \mathcal{B}(x) \subseteq B(x, \varepsilon)
  \subseteq I(x) \subseteq V \).

  We can summarize what we've found here:
% https://q.uiver.app/#q=WzAsNCxbMCwwLCJWIl0sWzIsMCwiQihWKSBcXHN1YnNldGVxIFxcbWF0aGNhbHtCfV9cXG1hdGhiYntRfSJdLFsxLDAsIkYiXSxbMywwXSxbMCwyLCJJIl0sWzAsMSwiQiIsMix7ImN1cnZlIjoyfV0sWzEsMCwiQl57LTF9IiwyLHsiY3VydmUiOjN9XV0=
\[\begin{tikzcd}[ampersand replacement=\&]
	V \& F \& {B(V) \subseteq \mathcal{B}_\mathbb{Q}} \& {}
	\arrow["I", from=1-1, to=1-2]
	\arrow["B"', curve={height=18pt}, from=1-1, to=1-3]
	\arrow["{B^{-1}}"', curve={height=18pt}, from=1-3, to=1-1]
\end{tikzcd}\]
  We also introduced the mapping \( B^{-1} \), taken as any right inverse of \( B
  \), i.e. a mapping such that:
  \[
    b = B(B^{-1}(b)), \forall b \in B(V) \iff B \circ B^{-1} = Id
  .\] 

  Here, consider the composition of \( B^{-1} \) and \( I \). It maps a subset
  \( B(V) \)
  of \( \mathcal{B}_{\mathbb{Q}} \), the set of all rational intervals, which is
  countable, to a subset \( F' = I(B^{-1}(B(V))) \) of the family \( F \). For
  every \( x \in V \), \( B(x) \in B(V) \). Then, letting \( x' = B^{-1}(B(x))
  \) (which in general, does not equal \( x \)), we have:
  \[
    B(x) = B(x') \implies x \in B(x) = B'(x) \subseteq I(x') = I(B^{-1}(B(x)))
    \in F'
  .\] 
  Hence, \( x \) must be in the union of all sets in \( F' \), a countable
  subset of \( F \).

  Now, for the general case, first observing that all non-trivial intervals can
  be written as an (countable, but that doesn't matter here) union of open
  intervals. Specifically, let \( I \) be a non-trivial interval, then there
  exists a family \( o(I) \) of open intervals such that:
  \[
    I = \bigcup_{J \in o(I)} J
  .\] 

  Now, consider the family \( F' = \bigcup_{I \in F} o(I) \). We have proven
  that there exists a subset \( K' \subseteq F' \) such that:
  \[
    \bigcup_{I \in K'} I = \bigcup_{I \in F'} I
  .\] 
  Now, noting that every intervals in \( K' \) originates from an interval of \(
  F\), or more specifically, there exists an "inverse" mapping \( o^{-1} \) such
  that:
  \[
    \forall I \in K', \exists o^{-1}(I) \in F, I \in o(o^{-1}(I))
  .\] 
  By how we constructed \( o(I) \), we must have \( I \subseteq o^{-1}(I) \) for
  all \( I \subseteq K' \).
  Hence, we have:
  \[
    \bigcup_{I \in F'} I =
    \bigcup_{I \in K'} I \subseteq \bigcup_{I \in K} o^{-1}(I) = \bigcup_{I \in
    o^{-1}(K')} I
    \subseteq
    \bigcup_{I \in F'} I
  .\] 
  Taking \( K = o^{-1}(K') \subseteq F \), we can conclude that:
  \[
    \bigcup_{I \in K} I =
    \bigcup_{I \in F} I
  ,\] with some countable \( K \subseteq F \).
\item 
  \label{2D5}
  Simply using Theorem 2.71 in MIRA (\( a \implies c \)), with the closed sets:
  \[
    F'_{k} = \bigcup_{n = 1}^{k} F_{k}
  .\] 
\item 
  \begin{enumerate}[label=(\alph*)]
    \item 
  If for every \( \varepsilon > 0 \),
  there exists some open set \( G \) such that:
  \[
    |G \setminus A| + |A \setminus G| < \frac{\varepsilon}{2} \implies |A
    \setminus G|, |G \setminus A| < \frac{\varepsilon}{2}
  .\] From the definition of outer measure, there exists some open interval
  cover \( I_{1}, I_{2}, \ldots  \) of \( A \setminus G \), with outer measure
  less than \( \frac{\varepsilon}{2} \). Let \( G' \) be the union of all \(
  I_{k} \)'s, then \( G' \) must be open. Then, consider the set \( G \cup G'
  \), which is a superset of \( A \), and:
  \[
    |(G \cup G') \setminus A| \le |G \setminus A| + |G' \setminus A| \le |G
    \setminus A| + \frac{\varepsilon}{2} < \varepsilon
  .\] 
  From Theorem 2.71 in MIRA, \( A \) is Lebesgue measurable.

  \item
  Now, assuming \( A \) is Lebesgue measurable, then there exists a
  sequence of open sets \( H_{n} \) all containing \( A \), such that:
  \begin{equation}
    \label{eq:2d8-1}
    \lim_{n \to \infty} |H_{n} \setminus A| = 0.
  \end{equation}
  Now, let \( A_{m} = A \cap [-m, m] \) be a sequence of bounded sets that
  covers \( A \), and since \( A_{m} \) is Lebesgue measurable, for every \( m
  \), there must exists a sequence \( F^{m}_{k} \) of closed sets such that:
  \begin{equation}
    \label{eq:2d8-2}
    \lim_{k \to \infty} |A_{m} \setminus F^{m}_{k}| = 0, \forall m \in
    \mathbb{N}^{*} \implies \lim_{m, k \to \infty} (|A_{m}| \setminus
    |F^{m}_{k}|) = 0.
  \end{equation}
  Then, since \( H_{n} \) is open, it could be written as:
  \[
    H_{n} = \bigcup_{x \in H_{an}} B_{n}(x)
  ,\] for some open ball \( B_{n}(x) \) with radius depending on \( x \). Now,
  \( B_{n}(x), x \in H_{n} \) forms an open cover of \( F^{m}_{k} \), so there
  exists a finite subcover \( B_{n}(x), x \in H'_{n, m, k} \subseteq H_{n} \) (\(
  H'_{n, m, k}
  \) is finite for all \( n, m, k \in \mathbb{N}^{*} \)).

  Letting \( G_{n, m, k} = \bigcup_{x \in H'_{n, m, k}} B_{n}(x) \), we have:
  \begin{align*}
    0 \le |G \setminus A| + |A \setminus G| &\le 2|H_{n} \setminus F^{m}_{k}| \\
    &= 2(|H_{n}\setminus A| + |A \setminus A_{m}| + |A_{m} \setminus
    F^{m}_{k}|)
  .\end{align*}
  Letting \( k \to \infty, n, m \to \infty \) (in that particular
  order: \( \lim_{n,m \to \infty} \lim_{k \to \infty} \ldots  \)), and using the
  results from problems \ref{2A8} and \ref{2A9}, the limit on the RHS is
  \( 0 \). Unpacking this limit, we have for every \( \varepsilon > 0 \), there
  exists some \( M_{1}, M_{2}(n, m) \) such that \( \forall n, m > M_{1}, k > M_{2}(n,
  m), \)
  \[
    |A \setminus G_{n, m, k}| + |G_{n, m, k} \setminus A| < \varepsilon
  .\] 
  \end{enumerate}

\item Similarly to problem \ref{2D5}, one simply uses Theorem 2.71.
\item Using Theorem 2.71, if \( A \) is Lebesgue measurable, then there exists
  a sequence \( G_{n} \) of open sets containing \( A \) such that:
  \[
    \lim_{n \to \infty} |G_{n} \setminus A| = 0
  .\]
  Since outer measure is translation invariant, the sequence \( t + G_{n} \)
  satisfies:
  \[
    \lim_{n \to \infty} |(t + G_{n}) \setminus (t + A)| = 0
  .\] 
  From this, we can conclude that \( t + A \) is Lebesgue measurable.
\item Since \( B \) is Lebesgue measurable, there exists some sequence \( F_{n}
  \) of closed subsets of \( B \) such that \( \lim_{n \to \infty} |B \setminus
  F_{n}| = 0\), or equivalently \( \lim_{n \to \infty} |F_{n}| = |B| \). Then:
  \[
    |A \cup B| \ge |A \cup F_{n}| = |A| + |F_{n}| \to |A| + |B|
  ,\] as \( n \to \infty \). Combining this with the subadditivity of the outer
  measure, we have:
  \[
    |A \cup B| = |A| + |B|
  .\] 
\item \label{2D12}
  One side is trivial. We will focus on proving that \( A \) being
  measurable can be implied from \( |A|+|(b,c) \setminus A| = c - b \). If
  either \( b \) or \( c \) is \( \pm \infty \), the problem becomes trivial
  (RHS is \( \infty \), and subadditivity becomes additivity in this case). For
  the finite \( b, c \) case, we first prove the following lemma.

  \textbf{Lemma.} For every subset \( S \) of \( \mathbb{R} \), there exists a
  Borel set \( T \) such that \( S \subseteq T \) and \( |S| = |T| \). Note that
  here \( |T| = |S| + |T \setminus S| \) is not necessarily true, in the case
  that \( S \) may not be Lebesgue measurable.

  \textbf{Proof.}
  The \( |S| = \infty \) case is trivial (\( T = \mathbb{R} \)). If \( |S| \) is
  finite, then by definition of the outer measure, we have:
  \[
    |S| = \inf \left\{ \sum_{k = 1}^{\infty} \ell(I_{k}): I_{1},I_{2},\ldots
    \text{ is an open interval cover of } S \right\}  
  .\] 
  From this, for every \( \varepsilon = \frac{1}{m} > 0 \), there exists an open
  interval cover \( I^{m}_{1}, I_{2}^{m}, \ldots  \) of \( S \) such that:
  \[
    |S| \le \sum_{k = 1}^{\infty} \ell(I_{k}^{m}) < |S| + \frac{1}{m}
  .\] 
  Then, \( T \) can be constructed as follows:
  \[
    T = \bigcap_{m = 1}^{\infty} \bigcup_{k = 1}^{\infty} I^{m}_{k}
  .\] 
  Here, the outer measure of \( T \) must be at most \( |\bigcup_{k =
  1}^{\infty} I^{m}_{k}| < |S| + \frac{1}{m} \) for all \( m \in \mathbb{N}^{*}
  \), which could only happen when \( |S| = |T| \).

  \textbf{Going back}, consider some Borel set \( B \) that contains \( A \)
  such that \( |A| = |B| \). To make sure that \( B_{1} \subseteq (b, c) \), one can
  reassign \( B_{1} \coloneqq B \cap (b, c) \), which is Borel.

  Doing the same thing to \( (b, c) \setminus A \), one ends up with a Borel set \(
  B'_{2} \) containing \( (b, c) \setminus A \) with the same measure as \( (b,
  c) \setminus A \). But since \( |A| + |(b, c) \setminus A| = c - b = |B_{2}'|
  + |(b, c) \setminus B_{2}'|\), we must have \( |A| = |(b, c) \setminus B_{2}'|
  \). Denoting \( B_{2} = (b, c) \setminus B_{2}' \), then one sees that \(
  B_{2} \subseteq A \) and \( |B_{2}| = |A| \).

  We don't have \( |B_{1}| = |A| + |B_{1} \setminus A| \), but we can get around
  this using the set \( B_{1} \) defined above. We have:
  \[
    0 \le |B_{1} \setminus A| \le |B_{1} \setminus B_{2}| = |B_{1}| - |B_{2}| =
    0
  ,\] which implies \( |B_{1} \setminus A| = 0 \). From this, by Theorem 2.71,
  \( A \) must be Lebesgue measurable.
\item 
  One side is trivial, we will only prove that \( A \) being Lebesgue measurable
  is implied from the equalities.

  Using problem \ref{2D12}, the equalities are equivalent to the fact that \(
  A_{n} = A \cap [-n, n]
  \) is Lebesgue measurable, \( \forall n \in \mathbb{N}^{*} \).
  From that, for every \( n \in \mathbb{N}^{*} \), there exists a sequence \(
  F^{n}_{k}\) of closed subsets of \( A_{n} \) such that:
  \[
    \lim_{k \to \infty} |A_{n} \setminus F^{n}_{k}| = 0, \forall n \in
    \mathbb{N}^{*}
  .\] 
  Combining this with the result of problem \ref{2A8} and \ref{2A9}:
  \[
    \lim_{n \to \infty} |A \setminus [-n, n]| = 0
  ,\] we have:
  \[
    0 \le |A \setminus F^{n}_{k}| \le |A_{n} \setminus F^{n}_{k}| + |A \setminus
    [-n, n] \setminus F^{n}_{k}| \le |A_{n} \setminus F^{n}_{k}| + |A \setminus
    [-n, n]|
  .\] 
  Letting \( k \to \infty, n \to \infty \) (\( \lim_{n \to \infty} \lim_{k \to
  \infty}  \)), we have:
  \[
    \lim_{n \to \infty} \lim_{k \to \infty} |A \setminus F^{n}_{k}| = 0
  .\] 
  Hence, for every \( \varepsilon > 0 \), there exists some closed set \( F =
  F^{n}_{k} \subseteq A \) such that \( |A \setminus F^{n}_{k}| < \varepsilon
  \), so \( A \) must be a Lebesgue measurable set.
\item Doing the division yields:
  \begin{align*}
    \frac{1}{4} &= 0.\overline{02}_{3}\\
    \frac{9}{13} &= 0.\overline{200}_{3}
  .\end{align*}
\item Doing the division yields an approximate result of \( 0.2021221102 \),
  so it must not be in the Cantor set\footnote{It also has no other base-3
  expansions, but to be precise, one can just take the interval containing the
number that was removed in \( G_{4} \)}.
\item No.
\item The set \( \frac{1}{2}C \) is the set containing all numbers which has a
  base-3 expansion containing only 0s and 1s. Now, consider a real number \( r
  \in [0, 1] \), with base-3 expansion \( (0.s_{1}s_{2}s_{3}\ldots )_{3} \). We
  can explicitly construct the base-3 expansions of \( \frac{1}{2}x \) and \(
  \frac{1}{2}y \) as follows:
  \begin{align*}
    s_{k} = 0 &\implies x_{k} = y_{k} = 0\\
    s_{k} = 1 &\implies x_{k} = 1, y_{k} = 0\\
    s_{k} = 2 &\implies x_{k} = y_{k} = 1\\
    \text{with } x &= (0.x_{1}x_{2}\ldots )_{3}, y = (0.y_{1}y_{2}\ldots )_{3}
  .\end{align*}

  Needless to say, since \( C \subseteq [0, 1] \), we have \( \frac{1}{2}C +
  \frac{1}{2}C \subseteq \frac{1}{2}[0, 1] + \frac{1}{2}[0, 1] = [0, 1] \).
\item 
  From every number \( x \in C \), one construct the sequence \( c(x)_{n} \) as
  follows:
  \[
    c(x)_{n} = x \text{, but flipping the } n \text{-th digit after the
      "decimal" point}
  .\] 
  Here, flipping is defined as simply replacing the digit \( 0 \) by \( 2 \) and
  vice versa. We can easily see that \( c(x)_{n} \in C \) for all \( n \in
  \mathbb{N}^{*} \), and all elements in this sequence is distinct.

  This sequence converges to \( x \), since \( |x - x_{n}| = \frac{2}{3^{n}} \)
  for all \( n \in \mathbb{N}^{*} \).

  Now, assuming that the intersection of an open interval with \( C \) has at
  least one element \( x \), then since the interval is open, there must be some
  \( \varepsilon > 0 \) such that \( B(x, \varepsilon) \) is contained in this
  interval. Hence, there exists some \( N > 0 \) such that \( c(x)_{n} \in B(x,
  \varepsilon), \forall n > N\). Hence, the intersection must have infinitely
  many elements.
\item 
  We have \( \Lambda (x) + \Lambda(1 - x)= 1 \) (trivial to prove).
  From that, we have:
  \[
    \int_{0}^{1} \Lambda (x) = \int _{0}^{1} \Lambda (1 - x) = \frac{1}{2}
  .\] 
\item 
  \( \Lambda \left( \frac{9}{13} \right)  = \Lambda(0.\overline{002}_{3}) =
  0.\overline{001}_{2} = \frac{1}{7}\).

  \( \Lambda(0.93) = \Lambda(0.221\ldots _{3}) = 0.111_{2} = \frac{7}{8} \).
\item
  \( \Lambda^{-1} \left( \left\{ \frac{1}{3} \right\}  \right) = \Lambda^{-1}
  (\{0.\overline{01}_{2}\}  ) = \{ 0.\overline{02}_{3}\} = \left\{ \frac{1}{4}
  \right\}  \).

  For \( \Lambda^{-1} \left( \left\{ \frac{5}{16} \right\}  \right) =
  \Lambda^{-1}(\{0.0101_{2}\}  )  \), it is a little bit more complicated.
  Assuming that \( x = (0.s_{1}s_{2}s_{3}s_{4}\ldots )_{3} \) such that \(
  \Lambda(x) = 0.0101_{2} \), then:
\( s_{1} = 0, s_{2} = 2, s_{3} = 0 \). There are two cases for \( s_{4} \):
\begin{itemize}
\item If \( s_{4} = 2 \), then \( s_{5}=s_{6}=\ldots =0 \), so \( x = 0.0202_{3}
  = \frac{20}{81}\). This is equivalent to the case \( s_{4} = 1, s_{5}
  = s_{6} = \ldots  = 2\).
\item If \( s_{4} = 1 \), then the digits from \( s_{5} \) can be arbitrary.
  From that, \( x \) can be any value in the range \( [0.0201_{3}, 0.0202_{3}] \).
\end{itemize}

To conclude, \( \Lambda^{-1} \left( \left\{ \frac{5}{16} \right\}  \right) =
[0.0201_{3}, 0.0202_{3}] = \left[ \frac{19}{81}, \frac{20}{81} \right]  \).

\item Basically, the function preserves repeating decimal (although in different
  bases), so it maps rationals to rationals.

  However, the truncation behavior of inputs not in the Cantor set makes it able
  to map an irrational number not in \( C \) to a rational number.

\item 
  We can encode any \( k \)-ary string by a real number \( x \) with fractional
  parts \( (0.s_{1}s_{2}\ldots)_{k} \). Now, using this, one can encode any
  expression as a real number, and given any prefix, one can always find a
  suffix such that the expression becomes an arbitrary real number.

  More specifically, the \textbf{Conway base 13 function} uses 10 base-10
  digits, and symbols \( + \), \( - \) and \( . \) in its encoding.

\item 
  \begin{enumerate}[label=(\alph*)]
    \item 
  If \( A \) is Lebesgue measurable, then there exists some sequence \(
  F_{n} \subseteq A \) of closed sets such that:
  \[
    \lim_{n \to \infty} |A \setminus F_{n}| = 0 \implies \lim_{n \to \infty}
    |F_{n}| = |A|
  .\] 
  Hence, the inner measure of \( A \) is at least \( \sup_{n \to \infty} |F_{n}|
  = |A|\).

\item 
  First, we will limit ourself to some large Borel subset \( X \) of \( \mathbb{R} \)
  with finite measure. For any set \( A \subseteq X \) and closed, bounded
  subset \( F \subseteq A \subseteq X \), we have:
  \[
    |F| = |X| - |X \setminus F|
  .\] 
  Since \( X \setminus F \) is an open set\footnote{And thus, can be written as
  the union of countably many open intervals.} containing \( X  \setminus A \),
  the infimum of \( |X \setminus F| \) is \( |X \setminus A| \). From this, we
  have:
  \[
    m(A) = |X| - |X \setminus A|
  .\] 
  Assuming the inner measure is additive.
  Consider two arbitrary subsets \( A, B \subseteq X \). We have:
  \begin{align*}
    |A| + |B| - |A \cup B| &= 2|X| - |X \setminus A| + |X \setminus B| + |X| + |X
    \setminus (A \cup B)|\\
    &= m(X) - m(A^{c}) - m(B^{c}) + m(A^{c} \cap B^{c}) \\
    &= 0
  ,\end{align*} with \( A^{c} = X \setminus A, B^{c} = X \setminus B \).
  Hence, the outer measure is additive for subsets of \( X \). However, since
  the example of disproving additivity of the outer measure in MIRA is a bounded
  set, this can not hold true. Hence, the inner measure could not be a measure
  on all subsets of \( \mathbb{R} \).
\end{enumerate}

  
\end{enumerate}

% section Exercises 2D (end)

\section{Exercises 2E} % (fold)
\label{sec:Exercises 2E}

Here, instead of using the original definition for uniform continuity, we will
use my personal lim-sup definition: \( f_{n} \) converges uniformly to \(
f\) on \( X \) if and only if:
\[
  \lim_{n \to \infty} \sup_{x \in X} |f(x)-f_{n}(x)| = 0
.\] 
The sup term may be infinity, but it does not matter if it only happens for
small values of \( n \). However, if that happens for all large \( n \), then
the limit is \( \infty \) and the condition will not be fulfilled.

Note that here, the \( \lim \sup \) is not the limit superior (\( \limsup \)),
it's just a limit of a supremum term. The equivalence of the definitions can be
easily proven by using the definition of limits and supremums.

\begin{enumerate}[label=\textbf{2E.\arabic*}]
  \item The definition of uniform continuity is equivalent to:
    \[
      \lim_{n \to \infty} \sup_{x \in X} |f_{n}(x)-f(x)| = 0
    .\] 
    When \( X \) is finite, the supremum term here has an upper bound of:
    \[
      0 \le \sup_{x \in X} |f_{n}(x) - f(x)| \le \frac{1}{n(X)} \sum_{x \in X}
      |f_{n}(x)-f(x)| \to 0
    ,\] as \( n \to \infty \) (here, \( n(X) \) denotes the cardinality of \( X \),
    which is finite).

    Hence, by the squeeze theorem, the supremum term must converges to \( 0 \),
    which means that \( f_{n} \) converges uniformly to \( f \).

  \item Take \( f_{n}(x) = \frac{x}{n} \). We have \( \lim_{n \to \infty}
    f_{n}(x) = \lim_{n \to \infty} \frac{x}{n} = 0 \) for all \( x \in
    \mathbb{R} \) (and \( \mathbb{Z}^{+} \)), but \( f_{n} \) does not converge
    uniformly to \( f(x) = 0 \), since:
    \[
      \lim_{n \to \infty} \sup_{x \in \mathbb{Z}^{+}} |f(x)-f_{n}(x)| = \lim_{n
      \to \infty} \sup_{x \in \mathbb{Z}^{+}} \frac{x}{n} = \lim_{n \to \infty}
      \infty = \infty \neq 0
    .\]
  \item 
    Take \( f_{n}(x) \) as:
    \[
      f_{n}(x) = \frac{x}{\left( x + \frac{1}{n} \right) ^2}
    ,\] then:
    \[
      f(x) = \lim_{n \to \infty} f_{n}(x) = \begin{cases}
        0, &\text{ if } x = 0\\
        \frac{1}{x}, & \text{ otherwise}
      \end{cases}
    ,\] is unbounded.

  \item 
    To reiterate, \( f \) being uniformly continuous is equivalent to:
    \[
      \lim_{\delta \to 0^{+}} \sup_{\substack{x_{0} \in A\\x \in B(x_{0}, \delta)}}
      |f(x)-f(x_{0})| = 0
    .\] 
    Here, we have:
    \begin{align*}
      |f(x)-f(x_{0})| &\le |f(x)-f_{k}(x)| + |f(x_{0})-f_{k}(x_{0})| +
      |f_{k}(x)-f_{k}(x_{0})|\\
                      &\le 2\sup_{x' \in A} |f(x')-f_{k}(x')| +
                      |f_{k}(x)-f_{k}(x_{0})|
    .\end{align*}
    Taking the supremum yields:
    \[
      0 \le \sup_{\substack{x_{0} \in A\\x \in B(x_{0}, \delta)}} |f(x)-f(x_{0})| \le
      2\sup_{x' \in A} |f(x')-f_{k}(x')| + \sup_{\substack{x_{0} \in A\\x \in
      B(x_{0}, \delta)}} |f_{k}(x)-f_{k}(x_{0})|
    .\] 
    To continue, we will apply the squeeze theorem by taking the limit (in this
    particular order of the RHS expression as \( \delta \to 0^{+} \) and \( k
    \to \infty \):
    \[
      \begin{array}{c|c|c}
        & \sup\limits_{x' \in A} |f(x')-f_{k}(x')| & \sup\limits_{\substack{x_{0} \in
        A\\ x \in B(x_{0}, \delta)}}|f_{k}(x)-f_{k}(x_{0})|
        \\
        \hline
        \delta \to 0^{+} & \sup\limits_{x' \in A} |f(x')-f_{k}(x')| & 0\\
        \hline
        k \to \infty & 0 & 
      \end{array}
    .\] 
    \begin{quote}
      Note that here, the order of taking limits is very important, much like in
      multivariable calculus, e.g. \( \lim_{n \to \infty} \lim_{m \to \infty} f(m, n)
      \neq \lim_{(m, n) \to \infty} f(m, n) \).
    \end{quote}
    From this, we can conclude that:
    \[
      \lim_{\delta \to 0^{+}} \sup_{\substack{x_{0} \in A\\x \in B(x_{0}, \delta)}}
      |f(x)-f(x_{0})| = 0
    \] and \( f \) is uniformly continuous\footnote{This is a method that uses
    no \( \varepsilon-\delta \) in its proof, purely based on limits and their
  properties. This method will be used instead of the more traditional methods
  throughout this document where applicable.} on \( A \).

\item Take \( f_{n}(x) = \frac{x}{n} \) with the pointwise limit \( f(x) = 0 \).
  If \( A \subseteq \mathbb{R} \) is unbounded, then:
  \[
    \lim_{n \to \infty} \sup_{x \in A} |f_{n}(x) - f(x)| = \lim_{n \to \infty}
    \infty = \infty \neq 0
  ,\] which means that \( f_{n} \) cannot converges to \( f \) uniformly on \( A
  \).

  However, if Egorov's Theorem holds in the \( \mu (X) = \infty \) case, then we
  would have some set \( E \) satisfying \( \mu (X \setminus E) = 0 \) such that
  \( f_{n} \) converges uniformly to \( E \). Such a set \( E \) would have to
  be unbounded, since \( \infty = \mu (E) \le  |\sup E - \inf E| \), which
  contradicts with what we have above.
\item Define Borel sets \( A_{m, n} \) as:
  \[
    A_{m, n} = \{x \in X: f_{k}(x) > n, \forall k \ge m\} = \bigcup_{k \ge m}
    f_{k}^{-1}((n, +\infty))
  .\] 
  Then, since \( f_{k} \) converges pointwise to \( \infty \), \( \forall x \in
  X, M \in \mathbb{R}, \exists m \in \mathbb{N}^{*}, f_{k}(x) > M, \forall k \ge
  m\). In other terms, \( x \in A_{m, M} \). Hence, we can write:
  \[
    X = \bigcup_{m \in \mathbb{N}^{*}} A_{m, M}, \forall M \in \mathbb{R}
  .\] 
  We have the sequence of sets \( A_{1, M} \subseteq A_{2, M} \subseteq \ldots
  \) is increasing, so taking the limit yields:
  \[
    \lim_{m \to \infty} \mu (A_{m, M}) = \mu (X)
  ,\] which means, for every \( M \) and \( \varepsilon > 0 \),
  there exists some \( m = m(\varepsilon, M) \) such that \(
  \mu (A_{m, M}) > \mu (X) - \varepsilon\).

  Take \( E = \bigcap_{M = 1}^{\infty} A_{m_{M}, M} \), then:
  \( E = \bigcap_{M = 1}^{\infty} A_{m_{M}, M} \) then:
  \[
    \inf_{x \in E} f_{k}(x) \ge \inf_{x \in A_{m_{M}, M}} f_{k}(x) \ge M
  ,\] for some \( M \) satisfying \( k \ge m_{M} \). This implies:
  \[
    \lim_{k \to \infty} \inf_{x \in E} f_{k}(x) = +\infty
  ,\] and therefore \( f_{n} \) converges uniformly to \( \infty \), on \( E \).
  Finally, we just need \( \mu (X \setminus E) < \varepsilon \):
  \begin{align*}
    \mu (X \setminus E) &= \mu \left( \bigcup_{M = 1}^{\infty} (X \setminus
    A_{m_{M}, M}) \right)  \\
                        &\le \sum_{M = 1}^{\infty} (\mu (X) - \mu (A_{m_{M},
                        M}))\\
                        &\le  \sum_{M = 1}^{\infty} 2^{-M - 1}\varepsilon =
                        \frac{\varepsilon}{2} < \varepsilon
  ,\end{align*} if we let \( m_{M} = 2^{-M-1}\varepsilon \) from earlier.

\item If \( g_{n} \) converges uniformly, then we are done. We will only
  consider the reverse statement.
  
  By definition, \( g_{n} \) converges uniformly to \( g \) is equivalent to:
  \[
    \lim_{n \to \infty} \sup_{x \in F} |g(x)-g_{n}(x)| = 0
  .\] 
  Note that here, the sequence \( u_{n} = \sup_{x \in F} |g(x)-g_{n}(x)| \) is
  a decreasing, non-negative sequence, so it must converge to some limit \( L
  \). Here, the function \( |g - g_{n}| = g - g_{n} \) is a continuous function,
  as \( g \) and \( g_{n} \) are both continuous, and since \( F \) is closed
  and bounded, then \( g-g_{n} \) must attain some maximum on \( F \).
  
  Let \( x_{n} \) be
  the value that maximizes \( g(x)-g_{n}(x) \) on F, then the sequence \( x_{n}
  \) must have a convergent subsequence \( x_{n_{k}} \).
  For brevity, denote \( y_{k} = x_{n_{k}}, y = \lim_{k \to \infty} y_{k} \),
  \( f_{k} = g_{n_{k}} \) and \( f = g \).

  Now, we have:
  \begin{align*}
    0\le L &\le |f(y_{k}) - f_{k}(y_{k})|\\
    &\le |f(y_{k}) - f_{k_{0}}(y_{k})|\\
    &\le |f(y_{k})-f(y)| + |f(y)-f_{k_{0}}(y)| +
    |f_{k_{0}}(y)-f_{k_{0}}(y_{k})|
  ,\end{align*} for all \( k_{0} \le k \)
  Now, take the limit (in this particular order) as \( k \to \infty \) and \(
  k_{0} \to \infty \) of RHS:
  \begin{center}
    \begin{tabular}{c|c|c|c}
      & \( |f(y_{k})-f(y)| \) & \( |f(y)-f_{k_{0}}(y)| \) & \(
      |f_{k_{0}}(y)-f_{k_{0}}(y_{k})| \)\\
      \hline
      \( k \to \infty \) & 0 & \( |f(y)-f_{k_{0}}(y)| \) & 0\\
      \hline
      \( k_{0} \to \infty \) &  & 0 & 
    \end{tabular}
  \end{center}
  \begin{quote}
  Here, if we naively let \( k_{0} = k \to \infty \), then \(
  f_{k_{0}}(y)-f_{k_{0}}(y_{k}) = f_{k}(y)-f_{k}(y_{k}) \) only converges if \(
  f_{k}\) converges
  uniformly\footnote{\href{https://math.stackexchange.com/questions/182223}{https://math.stackexchange.com/questions/182223}}.
  The variable \( k_{0} \) here is used as a workaround for this particular
  problem.
\end{quote}
  Hence, we must have \( L = 0 \), which implies uniform continuity.

\item For every \( \varepsilon > 0 \), one can always pick a finite \( E
  \subseteq \mathbb{Z}^{+} \) satisfying \( |\mathbb{Z}^{+} \setminus E| <
  \varepsilon \). Then, every sequence of function that converges pointwise also
  converges uniformly on \( E \).
\item For every \( x_{0} \in F_{1} \), there exists some \( \varepsilon > 0 \)
  such that \( B(x_{0}, \varepsilon) \) does not intersect \( F_{2}, F_{3},
  \ldots , F_{n} \). Assuming that this is not true, then for every \(
  \varepsilon_{n} > 0 \), \( B(x_{0}, \varepsilon_{n}) \) will intersect at
  least one of the \( n-1 \) sets \( F_{2}, F_{3}, \ldots , F_{n} \). Since
  there are infinitely many \( \varepsilon_{n} \) as one let \( \varepsilon_{n}
  \to 0\), one of the sets must contain a sequence \( x_{1}, x_{2}, \ldots  \)
  such that \( x_{k} \in B(x_{0}, \varepsilon_{k}) \cap F_{m}, \forall  k \in
  \mathbb{N}^{*} \) and \( x_{n} \) converges to \( x_{0} \). But since \( F_{m}
  \) is closed, we must have \( x_{n} \in F_{m} \), which leads to \( F_{m} \)
  and \( F_{1} \) intersecting, which is a contradiction.
\item If \( F \) is not closed, then \( \exists x_{0} \in \partial F \setminus F
  \). Then, define \( f \) as:
  \[
    f(x) = \frac{1}{x - x_{0}}
  \] yields a continuous function on \( F \), but could not be extended to a
  continuous function on \( \mathbb{R} \) (since \( \lim_{x \to x_{0}}
  f(x)=\infty \)).

\item We can do the same as the above problem, but adding \( \sin \) to make \(
  f\) bounded:
  \[
    f(x) = \sin \left( \frac{1}{x - x_{0}} \right) 
  ,\] with some \( x_{0} \in \partial  F \setminus F \).
  Then, if we can extend \( f \) to a function \( g \) that is continuous on \(
  \mathbb{R}\), then:
  \[
    g(x_{0}) = \lim_{x \to x_{0}} g(x) = \lim_{x \to x_{0}} \sin \left(
    \frac{1}{x - x_{0}} \right) 
  .\] 
  But since the limit on RHS does not converge, we have a contradiction. Hence,
  \( \partial  F \setminus F \) has to be empty, and therefore \( F \) is
  closed.

\item \label{2E12}
  Consider a closed set \( C \subseteq [0, 1] \) with positive measure and empty
  interior. 
  \begin{quote}
  One such example of \(
  C\) is the dense Cantor set, but that's outside the scope of this problem as we
  only care about the existence of such sets.
  \end{quote}

  Take \( f \) as:
  \[
    f(x) = \chi_{C}(x) = \begin{cases}
      1, &\text{ if } x \in C\\
      0, & \text{ otherwise}
    \end{cases}
  .\] 
  For every \( B \subseteq \mathbb{R} \) such that \( |\mathbb{R} \setminus
  B| = 0\), \( B \) must contain some element \( x_{0} \) in \( C \) (otherwise \( C
  \subseteq \mathbb{R} \setminus B \) but \( \mu (C) > \mu (\mathbb{R} \setminus
  B) \)).

  Then, we will prove that \( f_{|B} \) is not continuous at \( x_{0} \). First,
  since \( x_{0} \in C \), we must have \( f(x_{0}) = 1 \). However, we can
  construct a sequence \( x_{n} \in B \setminus C \) that converges to \( x_{0}
  \), i.e. showing that \( B(x_{0}, \varepsilon) \cap B \setminus C \) is
  non-empty for all small \( \varepsilon > 0 \).

  Rewrite this set as \( B(x_{0}, \varepsilon) \cap (\mathbb{R} \setminus C)
  \setminus (\mathbb{R} \setminus B) \). The set \( G = B(x_{0}, \varepsilon)
  \cap (\mathbb{R} \setminus C ) \) is the intersection of two open sets,
  therefore open. \( G \) also contains \( x_{0} \), so it is non-empty and
  therefore has positive measure (since \( \exists \varepsilon' > 0 \) such that
  \( B(x_{0}, \varepsilon') \subseteq G \)). The set \( \mathbb{R} \setminus B
  \) has zero measure, so the set \( G \setminus (R \setminus B ) \) must have
  measure equal to that of \( G \), which means that \( G \setminus (\mathbb{R}
  \setminus B ) \) must be non-empty.

  Hence, by picking \( x_{n} \in B(x_{0}, \varepsilon_{n}) \cap B \setminus C
  \), for some \( \varepsilon_{n} \to 0 \), we have a sequence \( x_{n} \in B \)
  satisfying:
  \[
    \lim_{n \to \infty} f(x_{n}) = 0 \neq 1 = f(x_{0})
  .\] 
\item Consider a non-Borel set \( S \subseteq [0, +\infty) \). For every \( t
  \in \mathbb{R} \), define:
  \[
    f_{t}(x) = \begin{cases}
      1, & \text{ if } x = t \text{ and } t \in S\\
      0, &\text{ otherwise}
    \end{cases}
  ,\] then \( f = \chi_{S} \), which is not Borel measurable.
\item If \( |x-b_{k}| > 2^{-k} \) for all \( k \in \mathbb{N}^{*} \),
  then:
  \[
    f(x) = \sum_{k=1}^{\infty} \frac{1}{4^{k}|x-b_{k}|} < \sum_{n=1}^{\infty}
    \frac{1}{4^{k} \cdot 2^{-k}} = \sum_{k = 1}^{\infty} \frac{1}{2^{k}} = 1
  .\] 
  The condition is equivalent to \( x \in X = \mathbb{R} \setminus \bigcup_{k =
  1}^{\infty} [b_{k} - 2^{-k}, b_{k} + 2^{-k}] \). One can calculate the outer
  measure of \( X \) as:
  \[
    |X| = |\mathbb{R}| - \sum_{k = 1}^{\infty} 2^{1-k} = \infty - 2 = \infty
  .\] 
  From this, we have:
  \[
    |\{x \in \mathbb{R}: f(x)<1\}| \ge |X| = \infty
  .\] 
\item 
  Let \( f_{1}, f_{2}, \ldots  \) be a sequence of simple functions that
  converge to \( f \). Consider a function \( f_{k} \):
  \[
    f_{k} = \sum_{n = 1}^{m} c_{n}\chi_{A_{n}}
  ,\] with Lebesgue measurable sets \( A_{1}, A_{2}, \ldots , A_{m} \).

  Let \( B_{1}, B_{2}, \ldots , B_{m} \) be Borel sets satisfying \( A_{k}
  \subseteq B_{k} \) and \( |B_{k} \setminus A_{k}| = 0 \) for all \( k \in \{1,
  2, \ldots , m\}   \). Define \( g_{k} \) as:
  \[
    g_{k} = \sum_{n=1}^{\infty} c_{n}\chi_{B_{n}} 
  .\] 
  Now, let \( g \) be the pointwise limit of \( g_{k} \) as \( k \to \infty \),
  defined on the set \( E = \{x \in B: \lim_{k \to \infty} g_{k}(x) \text{
  exists}\}   \).

  Then, if \( f(x) \neq g(x) \) then either \( x \notin E \) or \( f_{k}(x) \neq
  g_{k}(x)\) for some \( k \in \mathbb{N}^{*} \). Since the former implies the
  latter, this can be reduced to:
  \[
    \{x \in B: f(x)\neq g(x)\}  \subseteq \bigcup_{k \in
    \mathbb{N}^{*}} \{x \in B: f_{k}(x)\neq g_{k}(x)\}  
  .\] 
  On the other hand, we have:
  \[
    |\{x \in B: f_{k}(x) \neq g_{k}(x)\}| \le \sum_{n = 1}^{m} |B_{n} \setminus
    A_{n}| = 0
  ,\] so we can conclude that:
  \[
    |\{x \in B: f(x) \neq  g(x)\}|   = 0
  .\] 
\end{enumerate}

% section Exercises 2E (end)

% chapter Measures (end)
