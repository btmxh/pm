%! TEX root = intro-optimization/main.tex
\chapter{General optimization problem} % (fold)
\label{chap:General optimization problem}

\section{Optimization problems in real life} % (fold)
\label{sec:Optimization problems in real life}

\subsection{Production planning problem} % (fold)
\label{sub:Production planning problem}

Consider a production company \( X \). Company \( X \) can produce many types of
products, denoted as \( P_{1}, P_{2}, \ldots , P_{m} \). But producing
these products is of course, not free. In order to produce all of these
products, the production pipeline needs some ingredients, which will be denoted
as \( I_{1}, I_{2}, \ldots , I_{n} \). All of the ingredients is stored in some
sort of storage, and there is only a limited amount of them.

\begin{quote}
  If the company can buy ingredients from external sources, one could either set
  the limit to \( \infty \) (if the cost of buying is not significant), or
  introduce a "money" ingredient, which represents the money needed to buy the
  actual ingredients. Hence, one can see that this model is extremely versatile
  for many types of production planning.
\end{quote}

Now, consider the production pipeline of the product \( P_{i} \). Assuming that
in order to produce an unit of \( P_{i} \), the company needs a total of \( a(i,
j) \) units of the ingredient \( I_{j} \). One can make things more neat by
introducing a matrix \( A \) such that \( A^{j}_{i} = a(i, j) \), and all of the
production formulas are encoded in a single matrix.

If the company wants to produce \( x^{i} \) units of \( P_{i} \), then it would
take \( A^{j}_{i}x_{i} \) units of \( I_{j} \). The amount of units for \(
I_{j}\) in order to produce all products would be the sum of all \(
A^{j}_{i}x^{i} \), which is simply the product \( A^{j}x \). One can even take
it a step further: \( Ax \) is the row vector, with \( (Ax)^{j} = A^{j}x \)
being the total amount of units of \( I_{j} \) that the company needs in order
to produce \( x^{i} \) products \( P_{i} \) for all \( i \).

Let \( b \) be a row vector holding all of the limits, i.e. \( b^{i} \) being
the maximum units of \( I^{i} \) that the company could use in its production
pipeline. Then, the condition simply became \( Ax\le b \).

\begin{quote}
  Matrix comparison is simply defined as \( A > B \) if \( A - B \) only have
  positve entries. In other terms, \( A > B \) if \( A^{i}_{j} > B^{i}_{j} \)
  for all \( i \) and \( j \) such that the expression makes sense.
  Similarly, one can define the operations \( <, \ge, \le \). An important thing
  to note is this ordering of matrices is only a partial order.
\end{quote}

Another condition in this problem is \( x \ge  0 \), since the company could not
produce a negative amount of products. If some product \( P_{i} \) is
"inseparable", in the sense of that there could not be a decimal amount of \(
P_{i} \), then \( x^{i} \) needs to be a natural number. However, we will not
consider that possibility here.

Now, we can discuss the thing we want to optimize. As a company, the first thing
we would want to optimize is profits. However, we don't have enough data to
calculate that, as we don't yet know the cost for the ingredients. Hence, we
would be interested in optimizing the second best thing, revenue. Denote \(
c_{i} \) as the selling price of an unit of \( P_{i} \), then the total revenue
is \( c_{i}x^{i} \), or simply \( cx \).

After all of that modeling, we arrive at a purely mathematical problem.
\begin{align*}
  \max\, &f(x) = cx\\
  \text{s.t.}\, &x \ge 0\\
              &Ax \le  b
.\end{align*}

This problem is the \textbf{standard form linear program}, which will be
discussed in the linear programming chapter.

% subsection Production planning problem (end)

\subsection{House pricing prediction} % (fold)
\label{sub:House pricing prediction}

We are given a dataset, including many houses \( H^{1}, H^{2}, \ldots, H^{n} \).
For each house \( H^{i} \), we know its properties, encoded as a column vector:
\( X^{i}_{j}\) is the \( j \)-th property of the \( i \)-th house \( H^{i} \).
Here, we encoded all of the properties in a single matrix, like in the
above problem.

We also know the price of the house \( H^{i} \), which will be stored in the row
vector \( y \), with \( y^{i} \) denoting the price of the house \( H^{i} \).

Then, consider the prediction function \( f(x) = xw \), with \( x \) being the
column vector containing the predicted house's properties. If \( f \) is
100\% correct, then we would expect \( f(X_{i}) = X_{i}w = y_{i} \) for all
indices \( i \). However, when the number of properties is less than the number
of sample houses, this is generally not the case. Hence, one would consider the
errors \( e^{i} = y^{i} - X^{i}w \), or one can rewrite this as \( e = y - Xw
\), and try to "minimize" this vector. In order to do so, one would need to
"combine" all of the components of the vector into a single number, which could
be carried out using some \textbf{norm}. The \( L^{2} \)-norm is regularly used
here, since it has a very valuable property of being continuous\footnote{The
norm \( g(x) =\sqrt{x^{T}x}  \) may not be continuous at \( x = 0 \), but one
could simply consider its square \( g(x)^2= x^{T}x \), which is more efficient
to compute and is continuous everywhere.}.

After all of that, once again, we arrive at a purely mathematical problem.
\begin{align*}
  \min\, &g(w) = (y-Xw)^{T}(y-Xw)\\
  \text{s.t.}\, &w \in \mathbb{R}^{n}
.\end{align*}

\begin{quote}
  One could even go further by expanding \( g(w) \).
  \begin{align*}
    g(w) &= (y-Xw)^{T}(y-Xw)\\
         &= (y^{T}-w^{T}X^{T})(y-Xw)\\
         &= y^{T}y - y^{T}Xw - yw^{T}X^{T} + w^{T}X^{T}X^{T}w\\
         &= w^{T}X^{T}Xw - 2y^{T}Xw + \text{const, (since \( y^{T}Xw = yw^{T}X^{T} =
         \langle y, Xw \rangle \))} 
  .\end{align*}

  We will stop here, because we almost accidentally solved the \textbf{linear
  regression} problem.
\end{quote}

Another note is despite \( L^2 \)-norm is very nice to work with, the problem
utilizing the \( L^{1} \)-norm tends to have better results (especially when
there are outliers).

% subsection House pricing prediction (end)

% section Optimization problems in real life (end)

\section{General optimization problem} % (fold)
\label{sec:General optimization problem}

Consider the generalized optimization problem: \( \min f(x), x \in D \). In the
scope of this book, we will only consider \( f: D \to \mathbb{R} \) and \( D
\subseteq \mathbb{R}^{n} \).

The maximum variant of the problem, \( \max f(x), x \in D \) could be solved by
solving the minimum variant one: \( \min -f(x), x \in D \). Hence, we just need
to investigate the minimum variant problem.

The function \( f(x) \) has a fancy name, \textbf{objective function}, since our
\textit{objective} is to find an \( x \) such that \( f(x) \) is minimized.

The set \( D \) is called the \textbf{feasible set} or the \textbf{constrained
set}. An element \( x \in D \) is a \textbf{feasible solution},
\textbf{candidate solution}, or simply a \textbf{solution} of the problem. A
solution \( x \in D \) is \textbf{more optimal} than another solution \( y \in D
\) iff \( f(x) < f(y) \). Similarly \( x \) is \textbf{as optimal as} \( y \) if \(
f(x) = f(y)\), and \( x \) is \textbf{at least as optimal as} \( y \) if \( f(x)
\le f(y)\).

Consider a \( D' \subseteq D \), then a solution \( x \in D' \) is \textbf{\( D'
\)-optimal} iff \( x \) is at least as optimal as all \( x' \in D' \). If
equality only happens when \( x = x' \), then \( x \) is strictly \( D'
\)-optimal.

Since \( D \subseteq  \mathbb{R}^{n} \), we can consider Euclidean topology on \(
\mathbb{R}^{n} \) and use topological concepts like \textit{open sets, closed
sets, neighborhoods}, etc. If a solution \( x \) is (strictly) \( D' \)-optimal
for some neighborhood \( D' \) of \( x \), then one could say \( x \) is a
(strictly) locally optimal solution ((S)LOS) of the problem.

In the special case \( x \) is a (strictly) \( D \)-optimal solution, then one
would call \( x \) a globally optimal solution (GOS), or (in the strict case)
\textbf{the} strictly globally optimal solution (SGOS) of the problem.

We have these following trivial statements.
\begin{itemize}
  \item Strictly globally optimal solution for a problem, if it exists, is
    unique.
  \item A (strictly) globally optimal solution is a (strictly) locally
    optimal solution.
\end{itemize}

Denote the above problem as \( P \). Then, the set of all GOSes will be denoted
as \( \operatorname{Argmin}(P) \), or \( \operatorname{Argmin}\limits_{x \in D}
f(x) \). The SGOS, if it exists, is denoted as \( \operatorname{argmin}(P) \) \(
\operatorname{argmin}\limits_{x \in D} f(x) \).

\subsection{Basic topological concepts} % (fold)
\label{sub:Basic topological concepts}

\begin{definition}
A \textbf{topology} on a set \( X \) is a collection \( \Sigma \) of subsets of
\( X \), such that

\begin{itemize}
  \item \( \varnothing, X \in \Sigma \)
  \item If \( \mathcal{C} \subseteq \Sigma \), then \( \cup_{S \in
    \mathcal{C}} S \in \Sigma \).
  \item If \( \mathcal{C} \subseteq \Sigma \), then \( \cup_{S \in
    \mathcal{C}} S \in \Sigma \), with an additional condition that \(
    \mathcal{C} \) must be finite.
\end{itemize}

\( X \) and \( \Sigma \) could be grouped together to form a \textbf{topological
space} \( (X, \Sigma) \). A set \( S \in \Sigma \) is called as an \textbf{open
set} (or more precisely, \( S \) is relatively open on \( \Sigma \)).

A \textbf{metric} on a set \( X \) is a function \( d: X^2 \to  \mathbb{R} \)
such that

\begin{itemize}
  \item \( d(x, y) \ge 0, \forall x, y \in D \). Equality only holds if \( x = y
    \).
  \item \( d(x, y) = d(y, x), \forall x, y \in D \).
  \item \( d(x, y) \le d(x, z) + d(z, y), \forall x, y, z \in D \).
\end{itemize}

A set \( X \)  with a metric \( d \) on \( X \) could be grouped together to
form a metric space \( (X, d) \).
\end{definition}

Let \( (X, d) \) be a metric space. Denote the \textbf{open ball} with center
\( x_{0} \in X \) and radius \( \varepsilon > 0 \) with relative to \( X \)
as \( B_{X}(x_{0}, \varepsilon) = \{x \in X, d(x_{0},x) < \varepsilon\}  \). If
\( X \) can be implicitly inferred, we can write \( B(x_{0}, \varepsilon) =
B_{X}(x_{0}, \varepsilon) \).
Let \( \mathcal{B} \) be the set
of all open balls: \( \mathcal{B} = \{B(x_{0}, \varepsilon), x_{0} \in X,
\varepsilon > 0\}   \) and denote
\[
  \Sigma(X) = \bigcap \{\Sigma, \mathcal{B} \subseteq \Sigma, (X, \Sigma) \text{ is a
  top. space}\}
\] as the topology generated by \( \mathcal{B} \). Then, the topological space
\( (X, \Sigma) \) is called the \textbf{topological space generated from the
metric space} \( (X, d) \).

\begin{theorem}
\label{thr:Open sets in topological spaces generated by a metric space}
  \( A \) is an open set in the topological space \( (X, \Sigma) \) generated by
  the metric space \( (X, d) \) if and only if for every \( x \in A \), there
  exists some \( \varepsilon > 0 \) such that \( B_{X}(x, \varepsilon) \subseteq
  A\).
\end{theorem}

\begin{proof}
  Let \( \Sigma' \) be the set of sets \( A \) satisfying \( \forall x \in A,
  \exists \varepsilon > 0, B_{X}(x, \varepsilon) \subseteq A \).

  Now, we will prove \( \Sigma = \Sigma' \), by proving the following:
  \begin{itemize}
  \item \( \Sigma \subseteq \Sigma' \). We will prove \( B(x, \varepsilon) \subseteq
    \Sigma'\) for all \( x \in X \), \( \varepsilon > 0 \).
    Consider a ball \( B(x, \varepsilon) \) and \( y \in B(x, \varepsilon) \).
    Then, let \( \varepsilon' = d(x, y) \), we will prove that \( B(y, \delta)
    \subseteq B(x, \varepsilon) \) if \( \delta < \varepsilon - \varepsilon' \).
    This is true, assuming \( \exists z \in B(y, \delta) \) such that \( d(z, x)
    \ge  \varepsilon\), then \( \varepsilon \le  d(x, z) \le d(x, y) + d(y, z) <
    \varepsilon' + \delta\), which is a contradiction.

    Then, it is trivial to manually verify that \( (X, \Sigma') \) is a topological
    space, and because of the minimality of \( \Sigma \), we must have \( \Sigma
    \subseteq \Sigma'\).

  \item \( \Sigma' \subseteq \Sigma \). We will prove that every set \( B
    \in \Sigma' \) is in \( \Sigma \). Consider such sets \( B \), for every \(
    x \in B\), there exists some \( \varepsilon(x) > 0 \) such that \( B(x,
    \varepsilon(x)) \subseteq B \) for all \( x \in B \). Then, one can write \(
    B = \bigcap_{x \in B} B(x, \varepsilon(x))\), the union of (potentially
    infinitely) many open balls, which is in \( B \).
    Because \( \Sigma \) is a topology, \( B \)
    must be in \( \Sigma \), therefore \( \Sigma' \subseteq \Sigma \).
  \end{itemize}

  Hence, \( \Sigma = \Sigma' \).
\end{proof}

\begin{definition}[Limits]
\label{def:Limits}
  In a metric space \( (X, d) \), the sequence \( (x_{n})_{n \in \mathbb{N}} \)
  \textbf{converges} to a point \( x \in X \) if and only if \( \lim_{n \to
  \infty} d(x_{n}, x) = 0 \).

  In a topological space \( (X, \Sigma) \), the sequence \( (x_{n})_{n \in
  \mathbb{N}} \) \textbf{converges} to a point \( x \in X \) if and only if for
  every neighborhood \( N \) of \( x \), there exists a number \( M \) such that
  \( \forall  n > M, x_{n} \in N \). This definition is equivalent to the metric
  space definition for topological spaces generated by a metric space.

  If a sequence \( x_{n} \) converges to \( x \), then we denote \(
  \operatorname{Lim} x_{n} \) as the set of all limits of \( x_{n} \). If the
  limit is unique, then denote that limit as \( \lim_{n \to \infty} x_{n} \).
\end{definition}

\begin{proof}[Proof of equivalent definitions]
  If \( \lim_{n \to \infty} d(x_{n}, x) = 0 \), then for every \( \varepsilon >
  0\), there exists some \( M \in \mathbb{N} \) such that \( d(x_{n}, x) <
  \varepsilon \) for every \( n > M \). Since \( d(x_{n}, x) < \varepsilon \),
  \( x_{n} \in B(x, \varepsilon) \). Hence, for every neighborhood \( N \) of \(
  x_{n}\), 
  \begin{align*}
    \lim_{n \to \infty} d(x_{n}, x) = 0 &\iff \forall \varepsilon > 0, \exists
    N \in \mathbb{N}, \forall n > N, d(x_{n}, x) < \varepsilon\\
                                        &\iff \forall N = B(x, \varepsilon > 0),
                                        \exists M \in \mathbb{N}, \forall n > M,
                                        x_{n} \in N
  .\end{align*}
\end{proof}

Now, consider a sequence \( (x_{n}) \) that converges to both \( y_{1} \) and \(
y_{2}\) in a metric space. Then, \( d(y_{1}, x_{n}), d(y_{2}, x_{n}) \to  0 \)
as \( n \to  \infty \), and therefore \( 0 \le d(y_{1}, y_{2}) \le  d(y_{1},
x_{n}) + d(y_{2}, x_{n}) \to 0 \) as \( n \to  \infty\). Then, \( d(y_{1},
y_{2}) = 0 \) and \( y_{1} = y_{2} \). This means that, \textbf{in metric spaces,
the limit of a convergent sequence must be unique}.

Here are a few basic concepts in topology, extended from the notions of open and
closed sets, and many important properties of them.

\begin{theorem}
  For a topological space \( (X, \Sigma) \) generated from a metric space \( (X,
  d) \), these following statements hold true:
  \begin{itemize}
    \item Consider a set \( S \subseteq  X \). Then, the \textbf{complement} of \( S \)
      in \( X \), denoted as \( S^{c} \) is defined as \(
      S^{c} = X \setminus S \). The \textbf{boundary} of \( S \), denoted
      as \( \partial S \), is defined to be the set of all \( x \in X \) such
      that \( \forall \varepsilon > 0, B(x, \varepsilon) \) intersects both \(
      S \) and \( S^{c} \). Then, \( S \cap \partial S = \varnothing \)
      for all \( S \in \Sigma \).
    \item A set \( S \) is \textbf{closed} if and only if \( S^{c} \) is
      open. Then \( \partial S \subseteq S \) iff \( S \) is closed
    \item A set \( D \subseteq X \) is closed iff every convergent sequence \(
      (x_{n}) \) on \( X \), such that \( x_{n} \in D, \forall n \in \mathbb{N}
      \), converges to a point \( x \in D \).
    \item For a set \( S \subseteq  X \), its \textbf{closure}, \( \overline{S} =
      S \cup  \partial S\), is a closed set.
    \item Let \( S \) be a subset of \( X \). Then, \( x \in S \) is an
      interior point of \( S \) if there exists an open ball \( B(x,
      \varepsilon) \) which lies completely inside of \( S \), i.e. \( B(x,
      \varepsilon) \subseteq  S \). The set of all interior points of \( S \),
      denoted as \( \operatorname{Int} S \), is the \textbf{interior} of \( S
      \). Then for all \( S \subseteq  X \), \( \operatorname{Int} S = S \setminus
      \partial S\) is an open set.
  \end{itemize}
\end{theorem}

\begin{proof}
\begin{itemize}
  \item If \( \exists  x \in S \cap \partial S \), then every neighborhood
    \( B(x, \varepsilon) \) of \( x \) intersects \( S \) and \( S^{c} \).
    Because \( S \) is closed, \( B(x, \varepsilon) \subseteq  S \), which means
    that the intersection of \( B(x, \varepsilon) \) with \( S^{c} \) must be a
    nonempty subset of \( S \), which is a contradiction.
  \item Because of the "symmetry" in the boundary's definition, we trivially
    have \( \partial S = \partial (S^{c}) \). Consider a closed set \( S \),
    then let \( x \in \partial S \).
    Since \( S^{c} \) is open, \( \varnothing = S^{c} \cap \partial (S^{c}) =
    S^{c} \cap \partial S \), which means \( x \notin S^{c} \) or \( x \in S \).
    Hence, \( \partial S \subseteq S \). To prove the other direction, consider a
    set \( S \) that contains its boundary \( \partial S \). Its complement, \(
    S^{c}\) is open and does not intersect \( \partial S \). Take \( x \in
    S^{c} \), then \( x \notin \partial S \) and since \( \{x\}  \subseteq B(x,
    \varepsilon) \cap S   \), this intersection must not be empty. Since \( x
    \notin \partial S \), \( B(x, \varepsilon) \cap S^{c} = \varnothing \),
    which means that \( B(x, \varepsilon) \subseteq S
    \) for all \( \varepsilon > 0 \).
  \item Assuming \( D \) is not closed, then by definition, \( D^{c} \) is not open,
    which means that \( \exists x \in D^{c} \) such that for every \(
    \varepsilon_{n} > 0\), there exists some \( x_{n} \in B(x, \varepsilon_{n}) \)
    that is not in \( D^{c} \), or \( x_{n} \in D \). Then, if \( \varepsilon_{n}
    \to  0\) as \( n \to  \infty \), \( x_{n} \) is a sequence on \( D \) that
    converges to \( x \in D^{c} \), which is a contradiction.

    Let \( D \) be a closed set, then for every \( x_{n} \) that converges to \( x
    \), if \( x \in D^{c} \) then there exists a neighborhood \( N \) of \( x \)
    that does not intersect \( D \). Using the topological definition of limit, \(
    N\) must contain infinitely many \( x_{n} \), which is a contradiction to the
    fact that \( x_{n} \in D \) for all \( n \).
    \item Consider \( x \in \overline{S}^{c} \), then \( x \notin S, x \notin
    \partial S \). We will prove that there exists \( \varepsilon > 0 \) such
    that \( B(x, \varepsilon) \) does not intersect \( S \) and \( \partial S
    \). Assuming the contrary, \( B(x, \varepsilon) \) intersect \( \overline{S}
    \) for all \( \varepsilon > 0 \), then since \( x \notin \partial S \) and
    \( B(x, \varepsilon) \) intersects \( \overline{S}^{c} \) at at least one
    point \( x \), \( B(x, \varepsilon) \) must not intersect \( S \). Hence,
    \( B(x, \varepsilon) \cap \partial S \neq  \varnothing \) for all \(
    \varepsilon > 0 \).

    Take \( y \in B(x, \varepsilon) \), then take an neighborhood \( B(y,
    \varepsilon') \) of \( y \) inside \( B(x, \varepsilon) \). If \( y \in \partial
    X\), then \( B(y, \varepsilon') \) must intersect \( S \), which could not
    happen due to \( B(x, \varepsilon) \cap  S = \varnothing \). Hence, \( y
    \notin \partial S, \forall y \in B(x, \varepsilon) \), or \( B(x,
    \varepsilon) \cap  \partial S \), which is a contradiction. Therefore, \(
    \overline{S} \) is a closed set.

  \item Consider \( x \in \operatorname{Int} S = S \setminus \partial S \).
    Then, \( B(x, \varepsilon) \) intersects \( S \) (at at least a point \( x
    \)). If \( B(x, \varepsilon) \) intersects \( S^{c} \), then \( x \in
    \partial S \) and we have a contradiction. Therefore, \( B(x, \varepsilon)
    \) lies inside \( S \), which means that \( B(x, \varepsilon) \subseteq S \)
    for all \( \varepsilon > 0 \). Therefore, \( \operatorname{Int} S \) is
    open.
\end{itemize}
\end{proof}

\begin{definition}
  Let \( (X, \Sigma) \) be a topological space. A subset \( S \) of \( X \) is
  \textbf{compact} if and only if for every collection \( \mathcal{C} \subseteq
  \Sigma \) (open cover of \( S \)) s.t.\[
    S \subseteq  \bigcup_{s \in \mathcal{C}} s
  .\] , there exists a finite subcollection \( F
  \subseteq \mathcal{C} \) s.t. \[
    S \subseteq  \bigcup_{s \in F} s
  .\] 
\end{definition}

The topological definition of compactness is somewhat useless, at least on \(
\mathbb{R}^{n} \), so we would want an equivalent definition that is a little
bit more useful. That would be achieved with the help of the \textbf{Heine-Borel
theorem}, which would be proven below. First, we will prove some prerequisite
results.

\begin{theorem}
  In topological spaces generated by metric spaces, if a set \( S \) is compact
  then any sequence \( (x_n)_{n=1}^{\infty} \subset S \) has a convergent
  subsequence, or \( S \) is \textbf{sequentially compact}.
\end{theorem}

\begin{proof}
  If \( S \) is compact, then assuming \( (x_{n}) \) does not have a convergent
  subsequence. Then for every \( x \in S \), there exists some neighborhood \(
  U(x) \) s.t. \( (x_{n}) \cap  U(x) \) is finite.

  Assuming the contrary, \( (x_{n}) \cap U(x) \) is infinite for all
  neighborhood \( U(x) \) of \( x \). Then, for every \( \varepsilon > 0 \), one
  let \( U(x) = B(x, \varepsilon) \) and therefore there are infinitely many
  elements of the sequence \( x_{n} \) s.t. \( |x_{n} - x| < \varepsilon \).
  Pick \( i_{1} = 1 \), then for every \( n \), because \( x_{1 .. i_{n}} \) is
  finite, we can always pick some \( x_{i_{n+1}} \) from \( (x_{n}) \cap  B(x,
  \varepsilon_{n}) \setminus x_{1 .. i_{n}} \). Pick an arbitrary \(
  \varepsilon_{n} \to 0 \), for example \( \varepsilon_{n} = \frac{1}{n} \), we
  can construct the infinite sequence \( x_{i_{n}} \) that converges to \( x \),
  to the whole thing.

  We have \( S \subseteq \bigcup_{x \in S} U(x) \), which means that there is a
  finite set \( F \subseteq S \) such that \( S \subseteq \bigcup_{x \subseteq
  F} U(x) \). Because \( S \) is infinite\footnote{If \( S \) is finite, then
  the sequence \( x_{n} \) must have infinite repetitions, which is a free
  source for a convergent subsequence.}, there will be some \( x \in S \) such
  that \( U(x) \) contains infinitely many points in \( S \), which is a
  contradiction to what we have just proven.
\end{proof}

\begin{theorem}
  A sequence \( x_{n} \) is a \textbf{Cauchy sequence} wrt the
  metric space \( (X, d) \) if and only if
  \[
    \lim_{(m, n) \to  \infty} d(x_{m}, x_{n}) = 0
  .\]

  A set \( S \) is \textbf{totally bounded} if for any \( \varepsilon > 0 \), \( S \)
  could be covered by finitely many open balls with radius \( \varepsilon \).
  A set \( S \) is \textbf{complete} if every Cauchy sequence on \( S \)
  converges to a point \( x \) on \( S \).
  In topological spaces generated by metric spaces, if a set \( S \) is
  sequentially compact, then \( S \) is complete and totally bounded.

\end{theorem}

\begin{proof}
  Consider a sequence \( x_{n} \) in \( S \) with a convergent subsequence \(
  x_{i_{n}} \) that converges to \( x \). Then, we have \( d(x_{i_{n}},x) <
  \frac{\varepsilon}{2} \) and \( d(x_{i_{n}}, x_{m}) < \frac{\varepsilon}{2} \)
  for large \( m, n \). Using the triangle inequality, \( d(x_{m}, x) <
  \varepsilon \) for large \( m \), which means that \( x_{n} \) converges to \(
  x\). Hence, \( S \) is complete.

  If \( S \) is not totally bounded, then there exists some \( \varepsilon>0 \)
  such that any finite collection of open balls \( B(x_{1}, \varepsilon),
  B(x_{2}, \varepsilon), \ldots, B(x_{n}, \varepsilon)  \) could not cover \( S
  \). Then, one can pick \( x_{n+1} \) as a point in \( S \) but lies outside of
  all of the open balls. This process can continue indefinitely, yielding an
  infinite sequence \( x_{1}, x_{2}, \ldots  \). Hence, we have
  \( |x_{m} - x_{n}| > \varepsilon \) for
  all \( m, n \), which means that \( x_{n} \) and its convergent subsequence is
  not a Cauchy sequence. Because convergence implies Cauchy sequence, we arrive
  at a contradiction.
\end{proof}

\begin{theorem}
  In topological spaces generated by metric spaces, if a set \( S \) is complete
  and totally bounded, then \( S \) is compact.
\end{theorem}

\begin{proof}
  Assuming that \( S = S_{1} \) has an open cover \( U_{i} \) without any
  finite subcover.

  Consider the finite cover of \( S \) by open balls \( C_{i} = B(x_{i},
  \varepsilon) \), with \( i \) ranging from \( 1 \) to \( N \). Then, since \(
  N\) is finite, there must be some \( C_{i} \) that could not be covered by a
  finite subcover of \( U_{i} \). Denote this set as \( S_{2} \)

  Continuing on, we have a finite sequence of sets: \( S_{1}, S_{2}, \ldots  \)
  with \( S_{n\ge 2} \) being open balls with radius \( \varepsilon(n) \). Note
  that we can fine-tune and pick \( \varepsilon(n) \) as we want. Here, an
  arbitrary sequence that converges to \( 0 \) would work, for example \(
  \varepsilon_{n} = \frac{1}{n} \)

  Pick arbitrary \( x_{n} \)  in \( S_{n} \), then the sequence \( x_{n} \) is a
  Cauchy sequence, due to \( x_{m}, x_{n} \in S_{\min \{m,n\}  } \), which has
  arbitrary small radius. Hence, \( x_{n} \) converges to some point \( x \),
  which belongs to some \( U_{i} \). Since \( U_{i} \) is open, there must be
  some neighborhood \( B(x, \varepsilon) \subseteq U \). But since the radius of
  the open balls \( S_{n} \) converges to \( 0 \), at one point, \( S_{n} \)
  will lie inside \( B(x, \varepsilon) \) and therefore, inside \( U_{i} \).
  However, in the construction of \( S_{n} \), we have ensured that \( S_{n} \)
  does not have a finite subcover of \( U_{i} \), which contradicts with the
  fact that \( U_{i} \) alone can cover the whole \( S_{n} \) for large \( n \).
\end{proof}

Using the three theorems, one can see that in metric spaces, compact,
sequentially compact and complete-and-totally-bounded are equivalent.

Going back to \( \mathbb{R}^{n} \), we have the following theorem.

\begin{theorem}[Heine-Borel Theorem]
  Let \( S \) be a subset of \( \mathbb{R}^{n} \). Consider the Euclidean
  topological space defined above. If there exists some \( M \) such that \(
  d(x, 0) < M, \forall x \in S \), we say that \( S \) is \textbf{bounded}.

  Equivalently, a set \( S \) is bounded if there exists an open ball \( B \)
  s.t. \( S \subseteq B \).

  Then, \( S \) is compact if and only if \( S \) is closed and bounded.
\end{theorem}

\begin{proof}
  Since \( S \) is contained in the open ball \( B(0, M) \), we can just simply
  prove that \( B(0, M) \) has a finite cover with \( B(x, \varepsilon) \) for
  arbitrary \( \varepsilon > 0 \) and \( S \) is totally bounded. Such a cover
  is trivial (but tedious) to construct. The other direction is trivial since
  totally bounded implies bounded.

  Assuming that Cauchy convergence and regular convergence is equivalent on \(
  \mathbb{R}^{n} \)(a basic result in Calculus), let \( x_{n} \) be a
  convergent sequence on \( S \), then \( S \) is complete iff \(  \lim_{n \to
  \infty} x_{n} =x \in S \). Then, we need to prove that \( \lim_{n \to \infty}
  x_{n} \in S\) for all convergent sequences \( x_{n} \) iff \( S \) is closed.

  If \( S \) is closed, then assuming \( x \notin S \), then since \( S^{c} \)
  is open, there is a neighborhood of \( x \), which contains infinitely many
  elements in the \( x_{n} \) sequence, that lies completely outside of \( S \),
  contradiction.

  If every sequence \( x_{n} \in S \) converges to \( x \in S \), assuming \( S
  \) is not closed, then take some \( x \in S^{c} \) s.t. every neighborhood \(
  B(x, \varepsilon)\) of \( x \)  intersects with \( S \). Then let \( x_{n} \)
  be the intersection of \( B(x, \varepsilon_{n}) \) with \( S \), for some
  sequence \( \varepsilon_{n} \) that converges to \( 0 \), then \( 0 < d(x,
  x_{n}) < \varepsilon_{n} \) and therefore \( x_{n} \to  x \) with \( x_{n} \in
  S\) for all indices \( n \), which is a contradiction with the fact that \( S
  \) is complete.

  Therefore, closed-and-bounded is equivalent to complete-and-totally-bounded on
  Euclidean spaces, and by the previous three theorems, we have what we needed.
\end{proof}

% subsection Basic topological concepts (end)

\subsection{General condition for the existence of globally optimal solutions} % (fold)
\label{sub:General condition for the existence of globally optimal solutions}

First, we consider this theorem about open and closed sets.

\begin{theorem}
  Consider the following optimization problem, denoted as \( P \)
  \begin{align*}
    \min\, &f(x)\\
    \text{s.t.}\, &x \in D
  .\end{align*}

  Then \( \operatorname{Argmin}(P) \) is not empty if and only if the set \(
  f(D)_{+} = \{t \in \mathbb{R}, t \ge f(x), \forall x \in D\}   \) is closed
  (wrt \( \mathbb{R} \)) and
  has a finite infimum.
\end{theorem}

\begin{proof}
  If \( \exists x_{0} \in \operatorname{Argmin}(P) \), then \( f(D)_{+}
  \) has \( f(x_{0}) \) as a lower bound. Consider the set \( L = \{t \in
  \mathbb{R}, \exists x\in D, t < f(x)\}   \), which is the complement of
  the set \( f(D)_{+} \) on \( \mathbb{R} \), then we will prove \( L \) is
  open, and therefore \( f(D)_{+} \) is closed.

  Let \( t \in L \), then there exists some \( x \in D \) such that \( t <
  f(x) \). Take some \( \varepsilon > 0 \) such that \( t + \varepsilon <
  f(x) \), for example \( \varepsilon = \frac{f(x) - t}{2} \), then 
  \( \forall t' \in B(t, \varepsilon) \), \( t' < t + \varepsilon <
  f(x) \), and therefore \( t' \in L \). Hence, \( B(t, \varepsilon) \subseteq
  L\), which means that for all \( t \in L \), there is a neighborhood of \(
  t\) which lies completely inside \( L \), or in other words, \( L \) is an
  open set.
\end{proof}

\begin{theorem}
\label{thr:LSC compact condition}
  The function \( f(x) \) is \textbf{lower semi-continuous} at a point \( x_{0} \) if for
  some \( \varepsilon > 0 \), there exists a neighborhood \( B(x_{0}, \delta) \) of
  \( x_{0} \) s.t. \( f(x) - f(x_{0}) \ge  -\varepsilon, \forall x \in B(x_{0},
  \delta)\).

  If \( f(x) \) is lower semi-continuous, then the set \( L = f^{-1}((y,
  +\infty]) \) is open (wrt \( D \)). Hence, \( L^{c} = f^{-1}([-\infty, y]) \)
  is closed (wrt \( D \)).

  Furthermore, if \( D \) is compact, then the minimization problem
  \( P: \min f(x), x \in D \) always has a GOS.

  (All of these topological results hold true for the topological space \( (D,
  \Sigma) \))
\end{theorem}

\begin{quote}
  The function \( f(x) \) is \textbf{upper semi-continuous} at a point \( x_{0} \) if for
  some \( \varepsilon > 0 \), there exists a neighborhood \( B(x_{0}, \delta) \) of
  \( x_{0} \) s.t. \( f(x) - f(x_{0}) \le  \varepsilon, \forall x \in B(x_{0},
  \delta)\).

  If a function is both lower semi-continuous and upper semi-continuous at \( x_{0} \),
  then \(
  -\varepsilon \le  f(x) - f(x_{0}) \le  \varepsilon\), or equivalently \( |f(x)
  - f(x_{0})| \le  \varepsilon\) for all \( x \) in some
  neighborhood of \( x_{0} \). Then, \( f(x) \) is continuous at \( x_{0} \).

  Therefore, the concept of semi-continuity can be thought of as weaker
  conditions for continuity, however, semi-continuous functions that are not
  continuous are rare in general.
\end{quote}

\begin{proof}

  If \( f(x) = -\infty \) for some \( x \in D \), then \( x \) is a GOS. Hence,
  WLOG, assuming that \( f(x) \neq  -\infty \) for all \( x \in D \).

  First, we consider the following alternative definition for lower
  semi-continuity.

\begin{lemma}
  Denote \( \liminf_{x \to  x_{0}} f(x) = \lim_{\varepsilon \to 0^{+}} \inf f(
  B_{D}(x_{0}, \varepsilon) \setminus \{x_{0}\})    \) as the limit inferior of \(
  f(x) \) as \( x \) approaches \( x_{0} \) (\( D \) is the domain of \( f \)).
  
  Then, \( f \) is lower semi-continuous at point \( x_{0} \) iff \( \liminf_{x \to
  x_{0}} f(x) \ge  f(x_{0}) \)
\end{lemma}

\begin{proof}[Proof of lemma]
  \( f \) is lower semi-continuous at \(
  x_{0} \) iff \( f(x)\ge -\varepsilon \) for \( x \) in some neighborhood \(
  B_{D}(x_{0}, \varepsilon) \setminus \{ x_{0}\}  \) of \( x_{0} \). Note that
  \( f(x_{0}) \ge f(x_{0}) -\varepsilon \), then \( f(x) \ge f(x_{0}) -
  \varepsilon, \forall  \varepsilon > 0 \), or equivalently \( \inf
  f(B_{D}(x_{0}, \varepsilon) \setminus \{x_{0}\}  ) \ge -\varepsilon \).

  Denote LHS as \( g(\varepsilon) \), then \( g \) is increasing as \( t \)
  decreases to \( 0^{+} \). Then, let \( \varepsilon \to  0^{+} \), \(
  g(\varepsilon) \to  L \) (not necessarily finite) such that \( L \ge  g(x_{0})\).
\end{proof}

Consider the set \( L = f^{-1}((y, +\infty]) = \{x, f(x) > y)\}   \), we will prove
that \( L \) is open wrt \( D \).

Consider \( x_{0} \in L \), i.e. \( f(x_{0}) > y \). Since \( \liminf_{x \to
x_{0}} f(x) \ge f(x_{0}) > y \), by definition of limit inferior,
\( \lim_{\varepsilon \to  0^{+}} \inf
f(B(x_{0}, \varepsilon) \setminus \{x_{0}\}) > y   \), then there exists some
\( \varepsilon > 0 \) such that \( f(x) > y, \forall x \in B_{D}(x_{0},
\varepsilon) \) (note that we already have \( f(x_{0}) > y \)).

Then, \( B(x_{0}, \varepsilon) \) is a subset of \( L \), and hence \( L \) is
open. Because of this, its complement wrt \( D \), the set \( f(D)_{+} \) is closed.

Let \( t_{0} = \inf f(D)_{+} \), then \( f(D)_{+} \) is either \( (t_{0},
+\infty) \) (an open set, or closed if \( t_{0} = -\infty \)) or \( [t_{0},
+\infty] \) (a closed one).

To prove that \( t_{0} = -\infty \) could not happen, we will prove that \( f \)
is bounded on \( D \).

Consider the sequence \( x_{n} \) s.t. \( f(x_{n}) < t_{n} \) with an arbitrary
sequence \( t_{n} \) that converges to \( -\infty \). Then, \( x_{n} \) must
has a convergent subsequence \( x_{i_{n}} \) that converges to some \( x \in D \),
however: \( -\infty = \lim_{n \to
\infty} f(x_{n}) = \lim_{n \to \infty} f(x_{i_{n}}) = \liminf_{n \to \infty}
f(x_{i_{n}}) \ge f(\lim_{n \to \infty} x_{i_{n}}) = f(x) \). This implies \(
f(x) = -\infty \), which is a contradiction with \( f \) not yielding \( -\infty
\).

Hence, \( f(D)_{+} \) must be a set in the form of \( [a, +\infty] \) with \( a
= \sup f(D)_{+}\). Then, \( a \in f(D) \) and \( \operatorname{Argmin}(P)
= f^{-1}(a)\).
\end{proof}
A corollary of this theorem is the \textit{extreme value theorem}.

\begin{corollary}
  If \( f \) is continuous on compact \( D \subseteq \mathbb{R}^{n} \) and
  yields no infinities, then both the minimization and the maximization problems
  have GOS(es).
\end{corollary}

If the feasible set is not compact, we have to introduce another condition to
ensure the existence of a GOS.

\begin{theorem}
\label{thr:coercive condition}
  Let \( f \) be a lower semi-continuous function on nonempty closed
  \( D \subseteq \mathbb{R}^{n} \).
  Additionally, if \( f \) is upper coercive on \( D \), i.e.
  \[
    f(x) \to +\infty \text{ as } |x| \to  +\infty
  .\], then the problem has GOS(es).
\end{theorem}

\begin{proof}
  Consider \( x_{0} \in D \) and a set \( D' = \{x \in D, f(x) \le f(x_{0})\}
  \subseteq D \), then \( D' \) contains points that are at least as optimal as
  \( x_{0} \). Hence, we just have to prove that \( f \) has a \( D' \)-optimal
  solution, and that solution would be a GOS of the problem.

  Using the previous theorem, one would need to prove that \( D' \) is compact. By
  Theorem \ref{thr:LSC compact condition}, \( D' = f^{-1}([-\infty,
  f(x_{0})]) \) is closed.
  
  Then it suffices to show that \( D' \) is bounded. Assuming the converse, then
  for any \( M > 0 \), there exists some \( x
  \in D' \) such that \( |x| > M \). Letting \( M \to  +\infty \), then \( f(x)
  \to  +\infty\), which means that \( +\infty \le f(x_{0}) \), which could only
  mean that \( f(x_{0}) = +\infty \), which means that we have
  a bad choice for \( x_{0} \). We can try again with another \( x_{0}' \) such
  that \( f(x_{0}') \neq +\infty \), or if such \( x_{0}'\) does not exist, \(
  f(x) = +\infty \) on \( D \), and therefore all \( x_{0} \in D \) is are trivial
  GOSes of the problem.
\end{proof}

% subsection General condition for the existence of globally optimal solutions (end)

\subsection{Convex optimization problem} % (fold)
\label{sub:Convex optimization problem}

Let \( v_{1}, v_{2}, \ldots , v_{m} \in \mathbb{R}^{n} \) and \( \lambda \in
\mathbb{R}^{1\times m} \), then consider the matrix \( v \) s.t. \( v_{i} \) is
the \( i \)-th column of \( v \).

The matrix-vector multiplication \( v \lambda \) can be rewritten as \( v_{i}
\lambda^{i} \) or more traditionally \( \lambda^{i}v_{i} \). This expression is
a linear combination of vectors \( v_{1}, v_{2}, \ldots , v_{m} \) with respect
to weight vector \( \lambda \).

\begin{definition}
  Let \( v_{1}, v_{2}, \ldots , v_{m} \in \mathbb{R}^{n} \) and \( \lambda \in
  \mathbb{R}^{m} \). Arrange \( v_{1}, v_{2}, \ldots , v_{n} \) into a matrix \(
  A\) such that \( A_{i} = v_{i} \).

  Then, the \textbf{linear combination} (LC) of \( v_{1}, v_{2}, \ldots, v_{m}
  \) with respect to weight \( \lambda \) is the matrix-vector multiplication \(
  \lambda A\).

  An operation \( \mathcal{S} \) on \( v_{1}, v_{2}, \ldots , v_{m}, \lambda \)
  is called a \textbf{special linear combination operation} (SLCO) if it has a specific
  condition for \( \lambda \) (if \( \lambda \) does not satisfy this condition,
  the result is undefined) and it yields the exact same result as the LC of \(
  v_{1}, v_{2}, \ldots , v_{m} \) wrt weight \( \lambda \).

  Two SLCO are have the same kind if the conditions of the two operations are
  exactly the same.
\end{definition}

SLCOs are basically a restricted linear combination operation, which restricts
its \textbf{special linear span} of a set of vectors.

Convex combination is a SLCO. But to define convex combination, one would need
to define some other SLCOs.

\begin{definition}
  These following SLCOs are widely used in this section.
\begin{itemize}
  \item \textbf{Affine combination} is a SLCO with the condition of \( \sum_{i =
    1}^{m} \lambda_{i} = 1 \). Denote the operation as \(
    \mathcal{A}(A, \lambda) \)
  \item \textbf{Conical combination} is a SLCO with the condition of \( \lambda
    \ge 0 \) and \( \dim \lambda = 1 \). This operation is simply
    \( \lambda x \) for \( A = \{x\}   \).
  \item \textbf{Convex conical combination} is a SLCO with the condition of
    \( \lambda \ge  0\). Denote the operation as \( \mathfrak{C}(A, \lambda)
    \)
  \item \textbf{Convex combination} is a SLCO with the combined combination of
    affine combination and convex conical combination, the weight condition is \(
    \lambda \ge 0 \) \textit{and} \( \sum_{i = 1}^{n} \lambda_{i} = 1 \). Denote
    the operation as \( \mathcal{C}(A, \lambda) \). \textbf{Strictly convex
    combination} is a stricter SLCO than convex combination, requiring
    \( \lambda > 0 \) instead of \( \lambda \ge 0 \).

\end{itemize}
\end{definition}

With SLCOs out of the way, we have these new definitions.

\begin{definition}
  Consider a SLCO \( \mathcal{S} \), then:
  \begin{itemize}
    \item A \( \mathcal{S} \)-special linear space is a set \( V \) such that \( V \) is
      closed under \( \mathcal{S} \) of arbitrary weights. In other words, if \(
      A \subseteq V \), then \( \mathcal{S}(A, w) \in V \) for every weight \( w
      \) such that the combination is defined.

      In the case of \( \mathcal{S} \) is the affine, conical, convex conical
      and convex combination operations, then \( V \) is called to be an
      \textbf{affine space}, a \textbf{cone}, a \textbf{convex cone} and a
      \textbf{convex set}, respectively.

    \item Let \( V \) be a \( \mathcal{S} \)-special linear space. Then, the
      minimal linear space \( V' \) such that \( V \subseteq V' \) is called the
      \textbf{linear subspace associated with} \( V \). The dimension of \( V \)
      is defined to be the dimension of \( V' \).

    \item The \( \mathcal{S} \)\textbf{-special linear hull} of a set \( V \) is the
      minimal \( \mathcal{S} \)-special linear space containing \( V \).
      Alternatively, this set could be called as the \( \mathcal{S} \)
      \textbf{-special linear space generated by} \( V \).
\end{itemize}
\end{definition}

With the definitions out of the way, we have the following trivial properties.

\begin{itemize}
  \item Every linear space is a \( \mathcal{S} \)-special linear space.
  \item Affine spaces and convex cones are convex sets.
  \item Intersection of (potentially infinitely many) \( \mathcal{S} \)-special
    linear spaces is another \( S \)-special linear space.
  \item Conic spaces could be redefined as to be a set that is  closed under
    non-negative uniform scaling (if \( x \in
    V\) then \( kx \in V \) for some scalar \( k \ge 0 \)) and addition.
  \item An affine space \( V \) could be written as a sum of a point \( x \in V \)
    and a linear subspace \( L \): \( V = x + L = \{x + y, y \in L\}   \)
\end{itemize}

We will prove the last statement. Take \( x \in V \), we will prove that \( L = V -
x = \{v - x, v \in V\}  \) is a linear space. This is true, since for every
list  \( A \subseteq V \), \( A - x \subseteq V - x \) and \( \mathcal{L}(A   -
x, \lambda) \), the linear combination of \( A-x \) wrt weight \( \lambda \),
is equivalent to \( \mathcal{L}(A-x, \lambda)  = \mathcal{L}(A, \lambda) - x
\sum_{i} \lambda_{i} = \mathcal{L}\left([x] + A, [1 - \sum_{i}
  \lambda_{i}] + \lambda\right)\) - x. Note that the sum of all components of the
  vector \(\lambda + [1 - \sum_{i} \lambda_{i}] \) is \( 1 \), then the linear
  combination is an affine combination, which is in \( V \) because \( V \) is an
  affine space. Hence, we proved \( \mathcal{L}(A-x, \lambda) \) is in \( L = V - x
  \), which means that \( L \) is closed under linear combination.

% subsection Convex optimization problem (end)

Finally, we arrive at the definition of a convex function.

\begin{definition}
  A function \( f \) is convex on \( D \) if \( D \) is a convex space, and
  \[
    f(\mathcal{C}(A, \lambda)) \le \mathcal{C}(f(A), \lambda)
  .\] , for all finite \( A \subseteq \operatorname{dom} f \) and \( \lambda \)
  satisfying the convex weight conditions.

  Assuming that \( A \) has no duplicate points, then if
  equality only holds in the case \( \lambda = \mathbf{e_{i}} \) for some \(
  i\) (\( \mathbf{e_{i}} \)) is the \( i \)-th standard basis vector, i.e. the
  vector with all zeroes except the \( i \)-th entry being \( 1 \)), then \( f
  \) is \textbf{strictly convex} on \( D \).
\end{definition}

Equivalently, \( f \) is (finitely) convex iff \( f(\lambda x + (1-\lambda)y) \le \lambda
f(x) + (1 - \lambda)f(y) \) for all \( x, y \in A, \lambda \in [0, 1] \). This
can be proven using induction (wrt \( |A| \)).

One can expand the domain of convex sets, i.e. for every convex function \( D
\subseteq \mathbb{R}^{n} \to  \mathbb{R} \), there exists a function \( f^{*}:
\mathbb{R}^{n} \to  \mathbb{R} \cup  \{+\infty\}   \)
defined as:
\[
  f^{*}(x) = \begin{cases}
    f(x), &\text{ if } x \in D\\
    +\infty, & \text{otherwise}
  \end{cases}
,\] which is also convex. Hence, we will only consider convex functions \( f:
\mathbb{R}^{n} \to  \mathbb{R} \cup \{+\infty\}   \) from now on, since every
convex function on Euclidean space can be turned into this form. Also, we will
denote \( \overline{\mathbb{R}} = \mathbb{R} \cup  \{+\infty\}    \) to make
things less mouthful.

From a collection of fundamental convex functions like \( f(x) = x^{n} \) or \(
f(x) = \|x\|\), we can combine them to get more complicated one. The types of
combination are listed in the following theorem:

\begin{theorem}[Operations on convex functions]
\label{thr:Operations on convex functions}
  Let \( f_{i}: \mathbb{R}^{n} \to  \overline{\mathbb{R}} , i \in I \) be a
  (potentially-uncountable infinite) family of convex functions. Then, the
  following functions are convex:
  \begin{itemize}
  \item (Convex conical combination) \( f = \sum_{i \in I} \alpha^{i}f_{i}  \),
    with \( \alpha \ge 0 \) and \( I \) is finite.
  \item (Supremum and maximum) \( f(x) = \sup_{i \in I} f_{i}(x)
    \) (even when \(  I\) is uncountable infinite). When \( I \) is finite, \(
    f(x) = \max_{i \in I} f_{i}(x) \).
  \item (Precomposition with an affine function) Let \( g(x) = Tx + b \) be an
    affine function, then \( f = f_{i} \circ g \) is a convex function for all \( i
    \in I\).
  \item (Postcomposition with an monotonic increasing convex function) Let \(
    g: \mathbb{R} \to  \overline{\mathbb{R}}  \) be an increasing function,
    then \( f = g \circ f_{i} \) is a convex function for all \( i \in I \).
  \end{itemize}
\end{theorem}

\begin{proof}
  \begin{itemize}
  \item \( f(\mathcal{C}(A, \lambda)) = \alpha^{i}f_{i}(\mathcal{C}(A, \lambda))
    \le \alpha^{i}\mathcal{C}(f_{i}(A), \lambda) \le
    \mathcal{C}(\alpha^{i}f_{i}(A), \lambda)=\mathcal{C}(f(A), \lambda)\).
  \item A satisfying way to solve this is to use epigraphs. However, we will
    delay that until we have defined that. For now, we have \( f(\mathcal{C}(A,
    \lambda)) \le \sup_{i \in I} f_{i}(\mathcal{C}(A, \lambda) \le
    \sup_{i \in I} \mathcal{C}(f_{i}(A), \lambda) = \mathcal{C}(\sup_{i \in I}
    f_{i}(A),     \lambda) = \mathcal{C}(f(A), \lambda)\), which proves that \(
    f\) is convex.
  \item \( f(\mathcal{C}(A, \lambda)) = f_{i}(T\mathcal{C}(A,
    \lambda) + b) = f_{i}(\mathcal{C}(TA + b, \lambda)) \le
    \mathcal{C}(f_{i}(TA+b), \lambda) = \mathcal{C}(f(A), \lambda) \)
  \item \( f(\mathcal{C}(A, \lambda)) = g(f_{i}(\mathcal{C}(A, \lambda))) \le
    g(\mathcal{C}(f_{i}(A), \lambda)) \) (since \( g \) is increasing), and \(
    g(\mathcal{C}(f_{i}(A), \lambda)) \le \mathcal{C}(g(f_{i}(A)), \lambda) =
    \mathcal{C}(f(A), \lambda) \) (since \( g \) is convex).
  \end{itemize}
\end{proof}

After all of this theory about convex functions, we have the following theorem,
which states a valuable property about convex minimization problems.

\begin{theorem}
  Consider the minimization problem \( P: \min f(x), x \in D \), with \( D \) being
  a convex set, \( f(x) \) convex on \( D \). Then, if \( P \) has a LOS \(
  x_{0} \), then \( x_{0} \) is also a GOS. Moreover, if \( f \) is strictly
  convex, then \( x_{0} \) is the SGOS of \( P \).
\end{theorem}

\begin{proof}
  Let \( x_{0} \) be a \( B(x_{0}, \varepsilon) \)-optimal solution of \( P \).

  Then for any \( x \in D \) that is more optimal than \( x_{0} \),
  any convex combination of \( x_{0} \) and \( x \) would be more optimal than
  \( x_{0} \). We will prove that there is one such combination \( x' =
  \mathcal{L}(x_{0}, x, w) \in B(x_{0}, \varepsilon) \setminus \{x_{0}\}   \).

  We can easily prove that \( d(x_{0}, x') = \|x'-x_{0}\| = w\|x-x_{0}\| \), which
  can be arbitrary small (but still positive)
  s.t. \( w > 0 \), which means that one can always pick
  some \( w \) such that \( x' \in B(x_{0}, \varepsilon) \) and \( x' \) is more
  optimal than \( x_{0} \). Hence, there is no \( x \in D \) that is more
  optimal than \( x_{0} \).

  If \( f \) is strictly convex, then any convex combination of \( x_{0} \) with
  an \( x \neq  x_{0}  \) that is at least as optimal as \( x_{0} \)
  would be more optimal than both \( x \) and \( x_{0} \). Hence, such \( x \)
  could not exist, and \( x_{0} \) is the SGOS of the problem.
\end{proof}

% subsection Convex optimization problem (end)

% section General optimization problem (end)

% chapter General optimization problem (end)
